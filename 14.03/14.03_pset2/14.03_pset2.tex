\documentclass{article}
\usepackage{amsmath}
\usepackage{amssymb}
\usepackage{indentfirst}
\usepackage{color}
\usepackage{graphicx}
\usepackage{epstopdf}
\usepackage{geometry}
\geometry{left=2.5cm,right=2.5cm,top=2.5cm,bottom=2.5cm}

\title{14.03 Problem Set 2}
\author{Yijun Jiang}
%\email{yjjiang@mit.edu}
\date{\today}

\begin{document}
\maketitle

\section{Short Questions}
\subsection{Question 1}
\noindent\underline{Conclusion}: This statement is FALSE.

\noindent\underline{Explanation}: Brian cannot treat all goods to be inferior. An inferior good is demanded less when income increases. If all goods are inferior for Brian, when his income increases, he consumes everything less. Therefore, he cannot spend all of his budget. This violates non-satiation axiom. Brian must have at least one normal good on which he spends more than before.

\subsection{Question 2}
\noindent\underline{Conclusion}: This statement is FALSE.
\noindent\underline{Explanation}: We cannot compare utility between two individuals. This is because utility function is defined up to a monotonic transformation. It makes no sense to compare the pure value of Peter's utility with John's. Even if Peter's is larger, it does not mean anything.

\subsection{Question 3}
\noindent\underline{Conclusion}: This statement is TRUE.

\noindent\underline{Explanation}: \textcolor{red}{TODO}

\subsection{Question 4}
\noindent\underline{Conclusion}: This statement is TRUE.

\noindent\underline{Explanation}: For an indifference curve $U(x_1,x_2,\cdots,x_n)=U_0$, the slope in $x_1$-$x_2$ plane is given by
\begin{equation*}
	\left(\frac{\partial x_1}{\partial x_2}\right)_{x_3,\cdots,x_n}=-\frac{\partial U/\partial x_2}{\partial U/\partial x_1}
\end{equation*}
where the subscipts denote the variables that are held fixed in the partial derivative. Since $U(x_1,x_2,\cdots,x_n)$ is increasing for all $x_i$ ($i=1,2,\cdots,n$), $\partial U/\partial x_i>0$. Therefore, $(\partial x_1/\partial x_2)_{x_3,\cdots,x_n}<0$, which means that the indifference curve is downward sloping in $x_1$-$x_2$ plane. The same is true for every other pair of goods.

Moreover, marginal rate of substitution of $x_1$ for $x_2$ is
\begin{equation*}
	MRS_{x_1,x_2}=-\left(\frac{\partial x_2}{\partial x_1}\right)_{x_3,\cdots,x_n}
\end{equation*}
which decreases with $x_1$ due to the property of deminishing MRS. This means
\begin{equation*}
	\left(\frac{\partial^2x_2}{\partial x_1^2}\right)_{x_3,\cdots,x_n}>0
\end{equation*}
Similarly $\left(\frac{\partial^2x_1}{\partial x_2^2}\right)_{x_3,\cdots,x_n}>0$. Therefore, the indifference curve is bowed towards the origin. The same is true for every other pair of goods.

\subsection{Question 5}
\noindent\underline{Conclusion}: This statement is FALSE.

\noindent\underline{Explanation}: The relation $MRS_{x,y}=-dy/dx=p_x/p_y$ only holds for an interior solution. When the solution is binded by the non-negative constraint, i.e. $x\geqslant0$ and $y\geqslant0$, one of the goods has zero quantity. In this case, the bundle is still maximal (under non-negative constraint), though the relation $MRS_{x,y}=p_x/p_y$ no longer holds.

\section{Utility Maximization and Marshallian Demand Functions}
\subsection{Part 1}
The utility maximization problem under budget constraint is
\begin{align*}
	&U^*=\max_{x,y}\left(\frac{1}{4}\log x+\frac{3}{4}\log y\right)\\
	&\textup{s.t. }p_xx+p_yy\leqslant m
\end{align*}

And according to non-satiation axiom, this can be rewritten as
\begin{align*}
	&U^*=\max_{x,y}\left(\frac{1}{4}\log x+\frac{3}{4}\log y\right)\\
	&\textup{s.t. }p_xx+p_yy=m
\end{align*}

We can use Lagrange multiplier $\lambda$. The problem becomes
\begin{align*}
	L(x,y,\lambda)&=\frac{1}{4}\log x+\frac{3}{4}\log y+\lambda(m-p_xx-p_yy)\\
	\frac{\partial L}{\partial x}&=0\\
	\frac{\partial L}{\partial y}&=0\\
	\frac{\partial L}{\partial\lambda}&=0
\end{align*}

This leads to
\begin{align*}
	\frac{1}{4x}-\lambda p_x&=0\\
	\frac{3}{4y}-\lambda p_y&=0\\
	p_xx+p_yy&=m
\end{align*}

The solution is
\begin{align*}
	\lambda&=\frac{1}{m}\\
	x&=\frac{m}{4p_x}\\
	y&=\frac{3m}{4p_y}
\end{align*}

So the optimal consumption bundle is
\begin{equation*}
	(x^*,y^*)=\left(\frac{m}{4p_x},\frac{3m}{4p_y}\right)
\end{equation*}

\subsection{Part 2}
Her optimal utility is
\begin{align*}
	U^*&=\frac{1}{4}\log x^*+\frac{3}{4}\log y^*\\
	&=\frac{1}{4}\log\frac{m}{4p_x}+\frac{3}{4}\log\frac{3m}{4p_y}\\
	&=\log m-\frac{1}{4}\log 4p_x-\frac{3}{4}\log\frac{4}{3}p_y
\end{align*}

The dual problem is
\begin{align*}
	&E^*=\min_{x,y}(p_xx+p_yy)\\
	&\textup{s.t. }U(x,y)=\frac{1}{4}\log x+\frac{3}{4}\log y\geqslant U^*\\
	&\textup{where }U^*(p_x,p_y,m)=\log m-\frac{1}{4}\log 4p_x-\frac{3}{4}\log\frac{4}{3}p_y
\end{align*}

And similar to the previous part, the inequality constraint is in fact an equality for the optimal solution. 
\begin{align*}
	&E^*=\min_{x,y}(p_xx+p_yy)\\
	&\textup{s.t. }U(x,y)=\frac{1}{4}\log x+\frac{3}{4}\log y=U^*\\
	&\textup{where }U^*(p_x,p_y,m)=\log m-\frac{1}{4}\log 4p_x-\frac{3}{4}\log\frac{4}{3}p_y
\end{align*}

\subsection{Part 3}
The Marshallian demand function is just $x^*(p_x,p_y,m)$ and $y^*(p_x,p_y,m)$. Then the uncompensated price effect is
\begin{equation*}
	\frac{\partial x^*}{\partial p_x}=-\frac{m}{4p_x^2}
\end{equation*}

The income effect is
\begin{equation*}
	\frac{\partial x^*}{\partial m}x^*=\frac{1}{4p_x}x^*=\frac{m}{16p_x^2}
\end{equation*}

\subsection{Part 4}
From the Slutsky equation
\begin{equation*}
	\frac{\partial d_x}{\partial p_x}=\frac{\partial h_x}{\partial p_x}-\frac{\partial d_x}{\partial I}\frac{\partial E}{\partial p_x}
\end{equation*}
the substitution effect can be calculated by
\begin{align*}
	\frac{\partial h_x}{\partial p_x}&=\frac{\partial x^*}{\partial p_x}+\frac{\partial x^*}{\partial m}x^*\\
	&=-\frac{m}{4p_x^2}+\frac{m}{16p_x^2}\\
	&=-\frac{3m}{16p_x^2}
\end{align*}

\section{Indirect Utility and Expenditure Function}
\subsection{Part 1}
The Lagrangian is
\begin{equation*}
	L(x,y,\lambda)=u(x,y)+\lambda(m-p_xx-p_yy)
\end{equation*}

According to the envelope theorem,
\begin{align*}
	\frac{\partial v(p_x,p_y,m)}{\partial p_x}&=\left.\frac{\partial L(x,y,\lambda;p_x,p_y,m)}{\partial p_x}\right|_{x=x^*,y=y^*}=-\lambda x^*\\
	\frac{\partial v(p_x,p_y,m)}{\partial p_y}&=\left.\frac{\partial L(x,y,\lambda;p_x,p_y,m)}{\partial p_y}\right|_{x=x^*,y=y^*}=-\lambda y^*\\
	\frac{\partial v(p_x,p_y,m)}{\partial m}&=\left.\frac{\partial L(x,y,\lambda;p_x,p_y,m)}{\partial m}\right|_{x=x^*,y=y^*}=\lambda
\end{align*}

In the given example, $v(p_x,p_y,m)$ has explicit form,
\begin{equation*}
	v(p_x,p_y,m)=\log m-\alpha\log p_x-(1-\alpha)\log p_y
\end{equation*}

Therefore,
\begin{align*}
	\frac{\partial v(p_x,p_y,m)}{\partial p_x}&=-\frac{\alpha}{p_x}\\
	\frac{\partial v(p_x,p_y,m)}{\partial p_y}&=-\frac{1-\alpha}{p_y}\\
	\frac{\partial v(p_x,p_y,m)}{\partial m}&=\frac{1}{m}
\end{align*}

From the equations above, we get
\begin{align*}
	-\lambda x^*&=-\frac{\alpha}{p_x}\\
	-\lambda y^*&=-\frac{1-\alpha}{p_y}\\
	\lambda&=\frac{1}{m}
\end{align*}

The solution gives Marshallian demands,
\begin{align*}
	d_x(p_x,p_y,m)&=x^*=\frac{\alpha m}{p_x}\\
	d_y(p_x,p_y,m)&=y^*=\frac{(1-\alpha)m}{p_y}
\end{align*}

\subsection{Part 2}
From the indirect utility function, we can write $m$ in terms of $p_x,p_y$ and $v$.
\begin{align*}
	\log m&=v+\alpha\log p_x+(1-\alpha)\log p_y\\
	m&=\exp(v+\alpha\log p_x+(1-\alpha)\log p_y)\\
	&=p_x^\alpha p_y^{1-\alpha}e^v
\end{align*}

This gives the expenditure function,
\begin{equation*}
	E(p_x,p_y,v)=p_x^\alpha p_y^{1-\alpha}e^v
\end{equation*}

\subsection{Part 3}
The dual problem is
\begin{align*}
	&E=\min_{x,y}p_xx+p_yy\\
	&\textup{s.t. }u(x,y)=v
\end{align*}

The Lagrangian is
\begin{equation*}
	L(x,y,\lambda)=p_xx+p_yy+\lambda(v-u(x,y))
\end{equation*}

According to the envelope theorem,
\begin{align*}
	\frac{\partial E(p_x,p_y,v)}{\partial p_x}&=\left.\frac{\partial L(x,y,\lambda;p_x,p_y,m)}{\partial p_x}\right|_{x=x^*,y=y^*}=x^*\\
	\frac{\partial E(p_x,p_y,v)}{\partial p_y}&=\left.\frac{\partial L(x,y,\lambda;p_x,p_y,m)}{\partial p_y}\right|_{x=x^*,y=y^*}=y^*
\end{align*}

On the other hand, from the explicit form of $E(p_x,p_y,v)$, we get
\begin{align*}
	\frac{\partial E(p_x,p_y,v)}{\partial p_x}&=\alpha\left(\frac{p_y}{p_x}\right)^{1-\alpha}e^v\\
	\frac{\partial E(p_x,p_y,v)}{\partial p_y}&=(1-\alpha)\left(\frac{p_x}{p_y}\right)^\alpha e^v
\end{align*}

Therefore, Hicksian demands are
\begin{align*}
	h_x(p_x,p_y,v)&=x^*=\alpha\left(\frac{p_y}{p_x}\right)^{1-\alpha}e^v\\
	h_y(p_x,p_y,v)&=y^*=(1-\alpha)\left(\frac{p_x}{p_y}\right)^\alpha e^v
\end{align*}

\subsection{Part 4}
From Marshallian demands,
\begin{align*}
	p_xx^*&=\alpha m\\
	p_yy^*&=(1-\alpha)m
\end{align*}
We know that the best estimate is $\alpha=2/3$.

\subsection{Part 5}
Using the indirect utility function, her initial utililty is
\begin{equation*}
	v=\log200-\frac{2}{3}\log1-\frac{1}{3}\log1=\log200
\end{equation*}

Using the expenditure function, when $p_x$ is increased to $8$, her expenditure becomes
\begin{equation*}
	E^{new}=8^{2/3}\times1^{1/3}\times\exp(\log200)=800
\end{equation*}
which means that she needs an additional income of $800-200=600$ to maintain her walfare level.

\section{Consumer Theory}
\subsection{Case 1}
\noindent\underline{Conclusion}: It violates TRANSITIVITY.

\noindent\underline{Explanation}: According to Sally's comparison criterion, Dyson$^P$Toshiba (power), Toshiba$^P$Hoover (Weight), and Hoover$^P$Dyson (noise). This is a loop, which violates transitivity. In other words, from the first two comparisons, transitivity axiom asks that Dyson$^P$Hoover, but the third comparison makes the opposite statement.

\subsection{Case 2}
\noindent\underline{Conclusion}: It violates DIMINISHING MRS.

\noindent\underline{Explanation}: Suppose there is an auxiliary college Berklee2 that offers $1$ computer science class and $x$ music classes, and that Berklee2$^I$MIT. Since Berkeley$^P$MIT and utility is increasing with $x$, we have $x<8$. The midpoint bundle between MIT and Berkelee2 is $\textup{Mid}=(4.5,(x+1)/2)$. Harvard$^P$Mid since Harvard offers more courses in both subjects.

If diminishing MRS holds, Mid$^P$MIT, and thus Harvard$^P$MIT. But Jess has the opposite statement. Thus diminishing MRS is violated.

\subsection{Case 3}
\noindent\underline{Conclusion}: It violates CONTINUITY.

\noindent\underline{Explanation}: 

\end{document}
