\documentclass{article}
\usepackage[utf8]{inputenc}
\usepackage{amsmath}
\usepackage{amssymb}
\usepackage{indentfirst}
%\usepackage{forest}
%\usepackage{listings}
\usepackage{algorithm}
\usepackage[noend]{algpseudocode}
\usepackage{geometry}
\geometry{left=2.5cm,right=2.5cm,top=2.5cm,bottom=2.5cm}

\title{6.046/18.410 Problem Set 4}
\author{Yijun Jiang\\Collaborator: Hengyun Zhou, Eric Lau}
%\email{yjjiang@mit.edu}
\date{\today}

\begin{document}
\maketitle
\section{Learn to Fuel Wisely}
\subsection{Part (a)}
\noindent\underline{Description}: First run APSP on the graph $G=(V,E)$ with edge weights $l(u,v)$. Construct a new unweighed undirected graph $G'=(V,E')$, using all original vertices but newly defined edges, as follows: check $\delta(u,v)$ for all pairs. If $\delta(u,v)\leqslant K$, create an edge $(u,v)$. By this construction, all the island pairs that can be reached within one fill are connected. Finally run APSP on the new graph $G'$. The longest path among all pairs shortest paths in $G'$ gives the value $t$ we want.

\noindent\underline{Correctness}: We prove that in order to get from any vertex $u$ to any distinct vertex $v$, the smallest number of refills is the length of the shortest path $\delta'(u,v)$ in $G'$. Then the correctness of the algorithm follows.

Given a shortest path $p'_{uv}$ from $u$ to $v$ in $G'$, we can use the following refilling strategy: start by filling at $u$, then until reaching $v$, we always go to the subsequent vertex in $p'_{uv}$ and refill there. This can always be done since, by the construction of $G'$, we can reach from a vertex in $p'_{uv}$ to its subsequent vertex in one fill.

\section{Lazy Random Homework Solving}
\subsection{Part (a)}
\noindent\underline{Proof}: By induction on $k$. When $k=1$, suppose that there are $r_1$ friends working on this problem. $r_1\geqslant r$. The assignment becomes invalid when all these $r_1$ friends are assigned to TA Nirvan (N) or Kelly (K). The possibility is $P[1,\textup{invalid}]=2\times2^{-r_1}\leqslant 2^{1-r}=k2^{1-r}$. So the statement is true for the base case.

When $k>1$, suppose the statement holds for $k-1$. Therefore, $P[k-1,\textup{invalid}]\leqslant(1-k)2^{1-r}$. In other words, $P[k-1,\textup{valid}]\geqslant 1-(k-1)2^{1-r}$. When we add the $k$-th problem, we have
\begin{equation*}
	P[k,\textup{valid}]=P[k-1,\textup{valid}]P[1,\textup{valid}|\textup{assignment valid for previous $k-1$ problems}]
\end{equation*}
where the second factor is a conditional probability: the probability for the assignment to be valid for the one-problem case (the $k$-th problem), under the condition that the assignment is valid for the $k-1$-problem case (the privious $k-1$ problems). If we use the unconditional probability, the equality becomes an inequality because the unconditional probability is always no larger.
\begin{equation*}
	P[k,\textup{valid}]\geqslant P[k-1,\textup{valid}]P[1,\textup{valid}]
\end{equation*}
The inequality can be further relaxed,
\begin{align*}
	P[k,\textup{valid}]&\geqslant(1-(k-1)2^{1-r})(1-2^{1-r})\\
	&=1-k2^{1-r}+(k-1)2^{2-2r}\\
	&\geqslant1-k2^{1-r}
\end{align*}
which means $P[k,\textup{invalid}]\leqslant k2^{1-r}$. Therefore, by induction, the statements is true for all $k$. Larry fails to choose a valid assignment with probability at most $k2^{1-r}$.
\end{document}
