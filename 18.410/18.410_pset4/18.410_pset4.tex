\documentclass{article}
\usepackage[utf8]{inputenc}
\usepackage{amsmath}
\usepackage{amssymb}
\usepackage{indentfirst}
%\usepackage{forest}
%\usepackage{listings}
\usepackage{algorithm}
\usepackage[noend]{algpseudocode}
\usepackage{geometry}
\geometry{left=2.5cm,right=2.5cm,top=2.5cm,bottom=2.5cm}

\title{6.046/18.410 Problem Set 4}
\author{Yijun Jiang\\Collaborator: Hengyun Zhou, Eric Lau}
%\email{yjjiang@mit.edu}
\date{\today}

\begin{document}
\maketitle
\section{Learn to Fuel Wisely}

\section{Lazy Random Homework Solving}
\subsection{Part (a)}
\noindent\underline{Proof}: By induction on $k$. When $k=1$, suppose that there are $r_1$ friends working on this problem. $r_1\geqslant r$. The assignment becomes invalid when all these $r_1$ friends are assigned to TA Nirvan (N) or Kelly (K). The possibility is $P[1,\textup{invalid}]=2\times2^{-r_1}\leqslant 2^{1-r}=k2^{1-r}$. So the statement is true for the base case.

When $k>1$, suppose the statement holds for $k-1$. Therefore, $P[k-1,\textup{invalid}]\leqslant(1-k)2^{1-r}$. In other words, $P[k-1,\textup{valid}]\geqslant 1-(k-1)2^{1-r}$. When we add the $k$-th problem, we have
\begin{equation*}
	P[k,\textup{valid}]=P[k-1,\textup{valid}]P[1,\textup{valid}|\textup{assignment valid for previous $k-1$ problems}]
\end{equation*}
where the second factor is a conditional probability: the probability for the assignment to be valid for the one-problem case (the $k$-th problem), under the condition that the assignment is valid for the $k-1$-problem case (the privious $k-1$ problems). If we use the unconditional probability, the equality becomes an inequality because the unconditional probability is always no larger.
\begin{equation*}
	P[k,\textup{valid}]\geqslant P[k-1,\textup{valid}]P[1,\textup{valid}]
\end{equation*}
The inequality can be further relaxed,
\begin{align*}
	P[k,\textup{valid}]&\geqslant(1-(k-1)2^{1-r})(1-2^{1-r})\\
	&=1-k2^{1-r}+(k-1)2^{2-2r}\\
	&\geqslant1-k2^{1-r}
\end{align*}
which means $P[k,\textup{invalid}]\leqslant k2^{1-r}$. Therefore, by induction, the statements is true for all $k$. Larry fails to choose a valid assignment with probability at most $k2^{1-r}$.
\end{document}
