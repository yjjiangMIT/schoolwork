\documentclass{article}
\usepackage[utf8]{inputenc}
\usepackage{amsmath}
\usepackage{amssymb}
\usepackage{indentfirst}
\usepackage{forest}
\usepackage{geometry}
\geometry{left=2.5cm,right=2.5cm,top=2.5cm,bottom=2.5cm}

\title{6.046/18.410 Problem Set 1}
\author{Yijun Jiang}
%\email{yjjiang@mit.edu}
\date{\today}

\begin{document}
\maketitle
\section{Problem 1-1: 6.006 Review}
\subsection{Part (a)}
\subsubsection{Suppose $f(n)=\Theta(g(n))$, then $2^{f(n)}=\Theta(2^{g(n)})$}
This statement is FALSE.

Consider $f(n)=n^2$ and $g(n)=n^2+n$. Obviously we have $f(n)=\Theta(g(n))$. But $2^{g(n)}=2^n2^{f(n)}$, so it is impossible to find $C_1\geqslant0$ and $n_0\in\mathbb{N}$ such that $\forall n>n_0, C_1\cdot2^{g(n)}\leqslant 2^{f(n)}$.

In this case, $2^{g(n)}$ is NOT an asymptotic lower bound of $2^{f(n)}$. So the original statement is false.

\subsubsection{For any constants $a,b>0$, $af(n)+bg(n)=\Theta(\max{(f(n),g(n))})$}
This statement is TRUE.

Let $C_1=\min{(a,b)}$ and $C_2=a+b$. Clearly we have $C_1,C_2>0$ and
\begin{equation*}
0\leqslant C_1\max{(f(n),g(n))}\leqslant af(n)+bg(n)\leqslant C_2\max{(f(n),g(n))}
\end{equation*}
which means that $af(n)+bg(n)=\Theta(\max{(f(n),g(n))})$.

\subsubsection{Suppose $f(n)=o(1)$, then $f(n)g(n)=o(1)$}
This statement is FALSE.

Consider $f(n)=n^{-1}$ and $g(n)=n$. We have $f(n)=o(1)$, but $f(n)g(n)=1\neq o(1)$.

\subsubsection{Rank functions by order of growth}
The ordering (growth rate from large to small, i.e. $g_k=\Omega(g_{k+1})$ for $k=1,2,\cdots,11$) is given below. The proof can be found after this list.
\begin{align*}
g_1&=n!\\
g_2&=4^n\\
g_3&=2^n\\
g_4&=3^{\log^2n}\\
g_5&=(\log n)^{\log n}=g_6=n^{\log\log n}\\
g_7&=n^{10}\\
g_8&=n^3\\
g_9&=n\log n\\
g_{10}&=\sum_{k=1}^n\log k\\
g_{11}&=\log\log n\\
g_{12}&=100000^{100000000}
\end{align*}

$g_5=(\log n)^{\log n}$ and $g_6=n^{\log\log n}$ belong to the same equivalent class. In fact, they are equal. $g_9=n\log n$ and $g_{10}=\sum_{k=1}^n\log k$ belong to the same equivalent class. Each of the remaining functions is partitioned into its own equivalent class.

\noindent\underline{Proof:} To compare the first 7 functions in the list, we can take $\log$ first.
\begin{align*}
f_1&=\log{g_1}\approx n\log n-n\textup{ [Stirling approx.]}\\
f_2&=\log{g_2}=n\log 4\\
f_3&=\log{g_3}=n\log 2\\
f_4&=\log{g_4}=\log^2n\log 3\\
f_5&=\log{g_5}=\log{n}\log\log{n}\\
f_6&=\log{g_6}=\log{n}\log\log{n}\\
f_7&=\log{g_7}=10\log n
\end{align*}
Then it is clear that $g_1=\Omega(g_2),\cdots,g_6=\Omega(g_7)$. Moreover, we notice that $g_5=g_6$. The rest of the functions can be compared without taking $\log$. The only slightly tricky one is $g_{10}=\sum_{k=1}^n\log k$. Notice that $\log n\leqslant g_{10}\leqslant n\log n$, then the comparisons can be made.

Finally, I will prove $g_9=\Theta(g_{10})$. Since $y=\log x$ is concave, an integral can be used as the lower bound of the sum:
\begin{equation*}
g_{10}\geqslant(\ln 2)^{-1}\int_1^n\ln xdx=n\log n-\frac{n-1}{\ln 2}
\end{equation*}
On the other hand $g_{10}\leqslant n\log n$. So $g_{10}=\Theta(n\log n)$, and then $g_9=\Theta(g_{10})$.

\subsection{Recurrences}
\subsubsection{$T(n)=10T(n/3)+n^2$}
$T(n)=\Theta(n^{\log_310})$.

\noindent\underline{Proof:} Use the master theorem (case 1). $a=10$, $b=3$, $\log_ba=\log_310>2$. So $n^2=O(n^{\log_ba-\epsilon})$ for some $\epsilon>0$. Therefore, $T(n)=\Theta(n^{\log_310})$.

\subsubsection{$T(n)=9T(n/3)+n^2\log n$}
$T(n)=\Theta(n^2\log^2n)$.

\noindent\underline{Proof:} Use the master theorem (case 2). $a=9$, $b=3$, $\log_ba=2$. So $n^2\log n=\Theta(n^{\log_ba}\log n)$. Therefore, $T(n)=\Theta(n^2\log^2n)$.

\subsubsection{$T(n)=T(\sqrt{n})+\log n$}
$T(n)=\Theta(\log n)$.

\noindent\underline{Proof:}
\begin{align*}
T(n)&=T(\sqrt{n})+\log n\\
&=T(n^{1/4})+\log n+\frac{1}{2}\log n\\
&=\cdots\\
&=\Theta(1)+\left(1+\frac{1}{2}+\frac{1}{2^2}+\cdots\right)\log n\\
&=\Theta(\log n)
\end{align*}

\subsubsection{$T(n)=T(n/4)+T(n/2)+n$}
$T(n)=\Theta(n)$.

\noindent\underline{Proof:} The recurrence is linear. So the guess is $T(n)=\Theta(n)$. To prove this, use the substitution method. Suppose $C_1n\leqslant T(n)\leqslant C_2n$, where $C_1$ and $C_2$ are positive. Then for the lower bound we have
\begin{align*}
	T(n)&=T(n/4)+T(n/2)+n\\
	&\geqslant\left(\frac{1}{4}C_1+\frac{1}{2}C_1+1\right)n\\
	&=\left(\frac{3}{4}C_1+1\right)n\\
	&[desired]\geqslant C_1n
\end{align*}
Obviously, if $C_1\leqslant4$, the desired inequality holds. On the other hand, for the upper bound,
\begin{align*}
	T(n)&=T(n/4)+T(n/2)+n\\
	&\leqslant\left(\frac{1}{4}C_2+\frac{1}{2}C_2+1\right)n\\
	&=\left(\frac{3}{4}C_2+1\right)n\\
	&[desired]\leqslant C_2n
\end{align*}
Obviously, if $C_2\geqslant4$, the desired inequality holds. This completes the prove that $T(n)=\Theta(n)$.

\subsubsection{$T(n)=T(2n/3)+T(n/3)+n\log n$}
$T(n)=\Theta(n\log^2n)$.

\noindent\underline{Proof:} Use the recursion tree method. The tree is shown below (only the coefficients are written out). We notice that $2/3+1/3=1$, thus the coefficients in each level sum up to unity. This feature guarantees that there are $\Theta(n)$ leaves in the tree, eaching contributing constant runtime. Moreover, each level contributes $\Theta(n\log n)$ since the sum of coefficients in a level is unity. And there are $\Theta(\log n)$ levels (bounded between $\log_{3/2}n$ and $\log_3n$). In conclusion, we have
\begin{equation*}
	T(n)=\Theta(n)+\Theta(n\log n)\Theta(\log n)=\Theta(n\log^2n)
\end{equation*}

\begin{forest}
[1[2/3[4/9[$\cdots$]][2/9[$\cdots$]]][1/3[2/9[$\cdots$]][1/9[$\cdots$]]]]
\end{forest}

\end{document}
