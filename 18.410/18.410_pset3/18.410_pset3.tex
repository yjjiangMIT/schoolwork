\documentclass{article}
\usepackage[utf8]{inputenc}
\usepackage{amsmath}
\usepackage{amssymb}
\usepackage{indentfirst}
\usepackage{color}
\usepackage{forest}
\usepackage{fancyhdr}
\usepackage{algorithm}
\usepackage[noend]{algpseudocode}
\usepackage{geometry}
\geometry{left=2.5cm,right=2.5cm,top=2.5cm,bottom=2.5cm}

\title{6.046/18.410 Problem Set 3}
\author{Yijun Jiang\vspace{3pt}\\Collaborator: Hengyun Zhou, Eric Lau}
%\email{yjjiang@mit.edu}
\date{\today}

\pagestyle{fancy}
\lhead{Yijun Jiang}
\rhead{6.046/18.410 Problem Set 3}

\begin{document}
\maketitle

\section{Key Word Search}
\noindent\textbf{Convention}:

Here for a string, its leftmost digit is labeled as \#0. For example for $M=111000$, we have $M[0]=1$ and $M[5]=0$. A different indexing convention may lead to small differences in the following parts.

\subsection{Part (a)}
\noindent\textbf{Description}:

Scan from the 0-th to the $(r-s)$-th digit of $M$, digit by digit. For the $i$-th digit, check the matching between substring $M[i:i+s-1]$ and $S_1$, as well as between $M[i:i+s-1]$ and $S_2$. Use two counters to keep track of the number of successful matches of $S_1$ and $S_2$. When all the digits are checked, compare the two counters and output the more frequent substring.

~

\noindent\textbf{Correctness}:

This algorithm checks all the possible matches for $S_1$ and $S_2$ by directly comparing against the substring of $M$ at each location. Since the loop goes through all the $s$-digit substrings of $M$, it is guaranteed that the count gives the correct number of successful matches.

~

\noindent\textbf{Runtime}:

There are $r-s+1$ possible $s$-digit substrings. The comparison against each of them takes $O(s)$ time, since it involves $s$ single-bit comparisons. The overall runtime is thus $T(r,s)=O(s(r-s+1))=O(rs)$. The simplification to $O(rs)$ is based on the fact that $s<r$.

\subsection{Part (b)}
\noindent\textbf{Description}:

Define a function $f:\{0,1\}\mapsto\{-1,1\}$ that maps the binary digit $0$ to coefficient $-1$ and binary digit $1$ to coefficient $1$. Now construct the following polynomials
\begin{align*}
P(x)&=\sum_{i=0}^{r-1}f(M[i])x^{r-i-1}\\
Q_1(x)&=\sum_{i=0}^{s-1}f(S_1[i])x^i\\
Q_2(x)&=\sum_{i=0}^{s-1}f(S_2[i])x^i
\end{align*}

Notice that the polynomial for $M$ is decending in the power of $x$, i.e. the 0-th digit of $M$ corresponds to $x^{r-1}$, while the polynomials for $S_1$ and $S_2$ are ascending, i.e. the 0-th digit of $S_1$ or $S_2$ corresponds to $x^0$.

We claim that, in the polynomial $C_1(x)=P(x)Q_1(x)$, the coefficient for $x^i$, denoted by $c_{1,i}$, reflects the number of matched bits of $S_1$ in the substring $M[r-i-1:r+s-i-2]$, where $i=s-1,s,\cdots,r-1$. Specifically, this relation is given by
\begin{equation*}
\textup{\# of matched bits of $S_1$ at $M[r-i-1]$}=\frac{1}{2}(s+c_{1,i})
\end{equation*}

Replacing the index in $M$ by $i$, where $i=0,1,\cdots,r-s$, and focuing on unmatched bits,
\begin{equation*}
\textup{\# of unmatched bits of $S_1$ at $M[i]$}=\frac{1}{2}(s-c_{1,r-i-1})
\end{equation*}

Since we tolarate at most $e$ unmatches, if
\begin{equation*}
c_{1,r-i-1}\geqslant s-2e
\end{equation*}
we identify a successful match of $S_1$ at $M[i]$. Due to the range of $i$, the subscript $r-i-1$ runs from $s-1$ to $r-1$.

Obviously, the same is true for $S_2$. The proof is stated below. But before that, I will show an example. This helps clarify the indexing used here.

~

\noindent\textbf{Example}:

Let $M=110101$ and $S_1=100$. Then $r=6$ and $s=3$. According to the indexing convention, $M[0]=1,M[1]=1,M[2]=0,\cdots,M[5]=1$. This corresponds to a polynomial
\begin{equation*}
P(x)=x^5+x^4-x^3+x^2-x+1
\end{equation*}

The polynomial $Q_1(x)$, however, is ascending in the power of $x$.
\begin{equation*}
S_1(x)=1-x-x^2
\end{equation*}

Their product is
\begin{equation*}
C_1(x)=P(x)Q_1(x)=-x^7-2x^6+x^5+x^4-x^3+x^2-2x+1
\end{equation*}

Let $i=0$ for example. Number of unmatched bits at $M[0]$ can be calculated from $c_{1,r-i-1}=c_{1,5}$, i.e. the coefficient of $x^5$, which is 1.
\begin{equation*}
\textup{\# of unmatched bits of $S_1$ at $M[0]$}=\frac{1}{2}(s-c_{1,5})=\frac{1}{2}(3-1)=1
\end{equation*}
which is verified by the fact that $M[0:2]=110$ differs from $S_1$ by one digit.

~

\noindent\textbf{Correctness}:

Let $x,y\in\{0,1\}$. Notice that $f(x)f(y)=1$ if $x=y$, and $f(x)f(y)=-1$ if $x\neq y$. Therefore,
\begin{align*}
C_1(x)&=P(x)Q_1(x)\\
&=\sum_{i=0}^{r-1}\sum_{j=0}^{s-1}f(M[i])f(S_1[j])x^{r-i+j-1}\\
&=\sum_{k=0}^{r-s}x^{r-k-1}\sum_{j=0}^{s-1}f(M[k+j])f(S_1[j])\\
&=\sum_{k=0}^{r-s}(\textup{\# of matches at $M[k]$}-\textup{\# of unmatches at $M[k]$})x^{r-k-1}
\end{align*}
where a substitution $k=i-j$ has been made.

Let $n^m_{1,k}=\textup{\# of matched bits of $S_1$ at $M[k]$}$, and $n^u_{1,k}=\textup{\# of unmatched bits of $S_1$ at $M[k]$}$. From the equation above, it is clear that $c_{1,r-k+1}=n^m_{1,k}-n^u_{1,k}$. Since $n^m_{1,k}+n^u_{1,k}=s$, we have
\begin{equation*}
n^u_{1,k}=\frac{1}{2}(s-c_{1,r-k-1})
\end{equation*}

A successful match of $S_1$ has $n^u_{1,k}\leqslant e$. The criterion is thus
\begin{equation*}
c_{1,r-k-1}\geqslant s-2e
\end{equation*}

The same is true for $S_2$.

\subsection{Part (c)}
\noindent\textbf{Description}:

See the pseudocode.
\begin{algorithm}
\caption{Solving Eve's string matching problem in $O(r\log r)$}
\begin{algorithmic}[1]
\Procedure{StringMatching}{$M,S_1,S_2,e$}
\State{Calculate coefficients of $P(x),Q_1(x)$ and $Q_2(x)$}
\State{Call \textsc{FFT} to evaluate $P(x),Q_1(x)$ and $Q_2(x)$ on a collapsing set $A$ of $r+s-1$ points: $x_1,\cdots,x_{r+s-1}$}
\State{Calculate $C_1(x_1),\cdots,C_1(x_{r+s-1})$ and $C_2(x_1),\cdots,C_2(x_{r+s-1})$, where $C_1=PQ_1$ and $C_2=PQ_2$}
\State{Call \textsc{IFFT} to calculate $c_{1,i}$ and $c_{2,i}$, i.e. coefficients of $C_1(x)$ and $C_2(x)$, where $i=0,\cdots,r+s-2$}
\State{$count1,count2\gets0$}
\For{$i=s-1:r-1$}\Comment{Counting, only coefficients of $x^{s-1},\cdots,x^{r-1}$ matters}
	\If{$c_{1,i}\geqslant s-2e$}
		\State{$count1++$}
	\EndIf
	\If{$c_{2,i}\geqslant s-2e$}
		\State{$count2++$}
	\EndIf
\EndFor
\If{$count1>count2$}
	\Return{$S_1$}
\ElsIf{$count1<count2$}
	\Return{$S_2$}
\Else{}
	\Return{$S_1,S_2$}
\EndIf
\EndProcedure
\end{algorithmic}
\end{algorithm}

~

\noindent\textbf{Correctness}:

From the previous part, we know that $n^u_{\alpha,k}=\frac{1}{2}(s-c_{\alpha,r-k-1})$, where $\alpha=1,2$. Therefore, by looking at the coefficients of the product polynomials, we can determine the number of ``good matches", i.e. the number of $k$ such that $n^u_{\alpha,k}\leqslant e$. The pattern that has more good matches is a more frequent pattern. Notice that since $k=0,\cdots,r-s$, only the coefficients of $x^{r-k-1}$, in other words $x^{s-1},\cdots,x^{r-1}$, matters.

~

\noindent\textbf{Runtime}:

The first step, calculating coefficients for $P(x),Q_1(x)$ and $Q_2(x)$, costs $O(r)+O(s)+O(s)=O(r)$ time. The second step, FFT, costs $O((r+s)\log(r+s))=O(r\log r)$ time. The third step, evaluating $C_1(x)$ and $C_2(x)$ at $r+s-1$ points, costs $O(r+s)=O(r)$ time. The fourth step, IFFT, costs $O(r\log r)$ time. Finally, counting and comparing costs $O(r-s)=O(r)$ time. Therefore, the total runtime is $O(r\log r)$.

\section{Optical Fiber Network}
\subsection{Part (a)}
\noindent\textbf{Proof}:

By contradiction. Suppose $S$ differs from $T$ by more than one edges. Consider the set $T-S$ of all edges that are in $T$ but not in $S$. Let $e_T$ be the edge of minimum weight in $T-S$. Adding $e_T$ into $S$ creates a cycle $C_S$. $T$ cannot contain cycles, so $C_S\not\subseteq T$, which means that there must be some $e_S\in C_S$ such that $e_S\in S-T$.

Adding $e_S$ into $T$ creates a cycle $C_T$. By the same reasoning, there must be some $e'_T\in C_T$ such that $e'_T\in T-S$. Consider the spanning tree $T'=T-\{e'_T\}+\{e_S\}$. We have $w(T)<w(T')$ due to the uniqueness of $T$ as an MST. Therefore, $w(e'_T)<w(e_S)$. Moreover, since $e_T$ is chosen to be the lightest edge in $T-S$, $w(e_T)\leqslant w(e'_T)$. Therefore, $w(e_T)<w(e_S)$.

Consider the spanning tree $S'=S-\{e_S\}+\{e_T\}$. Then $w(S')<w(S)$. Since $S'$ only differs from $S$ by one edge, $S'\neq T$. Therefore, $w(T)<w(S')<w(S)$, contradicting the fact that $S$ is the second minimum spanning tree.

Consequently, when MST is unique, a second minimum spanning tree differs from MST by exactly one edge.

\subsection{Part (b)}
\noindent\textbf{Description}:

For convenience let $D[u,u]=\textup{NULL}$ and let its weight be $w(\textup{NULL})=-\infty$. The algorithm performs a depth-first traversal of $T$. Maintain $A$ as a subset of $V$ containing all the vertices that are visited. $A$ is initialized to contain only the root $r$. Maintain the property that $D[u_1,u_2]$ is determined for every $u_1,u_2\in A$. This is done by the following: whenever a new vertice $v$ is visited, calculate $D[u_i,v]=D[v,u_i]$ for all $u_i\in A$, before adding $v$ into $A$. This calculation is done by the equation below, where $w(u,v)$ denotes the weight of $(u,v)$.
\begin{equation*}
D[u_i,v]=\left\{\begin{array}{ll}D[u_i,v.father]&\textup{if }w(D[u_i,v.father])\geqslant w(v,v.father)\\(v,v.father)&\textup{otherwise}\end{array}\right.
\end{equation*}
which holds for all newly added $v$. $v\neq r$ since $r\in A$ at the beginning, so $v.father$ exists. Moreover, $v.father\in A$ when $v$ is visited since the traversal is depth-first, so $D[u_i,v.father]$ can be accessed.

The procedure ends when every vertex is visited. In the meantime, $D$ is complete filled.

The pseudocode contains more details.
\begin{algorithm}
\caption{Finding the longest edge on a unique path}
\begin{algorithmic}[1]
\Procedure{LongestEdge}{$T$}
\State{$A\gets[T.root]$}
\State{$D(T.root,T.root)\gets\textup{NULL}$}
\State{\Call{Depth-First}{$T.root,A,D$}}
\State{\Return{D}}
\EndProcedure

\Procedure{Depth-First}{$root,A,D$}
\If{$root.childNum=0$}
	\Return{}
\Else{}
	\For{$i=1:root.childNum$}
		\For{$u$ in $A$}
			\If{$w(D[u,root])\geqslant w(root,child[i])$}\Comment{$w(\textup{NULL})=-\infty$}
				\State{$D[u,root.child[i]]\gets D[u,root]$}
				\State{$D[root.child[i],u]\gets D[u,root]$}
			\Else{}
				\State{$D[u,root.child[i]]\gets(root,child[i])$}
				\State{$D[root.child[i],u]\gets(root,child[i])$}
			\EndIf
		\EndFor
		\State{$D[root.child[i],root.child[i]]\gets\textup{NULL}$}
		\State{$A.append(root.child[i])$}
		\State{\Call{Depth-First}{$root.child[i],A,D$}}
	\EndFor
	\State\Return{}
\EndIf
\EndProcedure
\end{algorithmic}
\end{algorithm}

~

\noindent\textbf{Correctness}:

Since $A$ is connected, for $u_1,u_2\in A$, all the edges of $T$ in the unique path connecting $u_1$ and $u_2$ lie in $A$. So $D[u_1,u_2]$ can be determined within $A$, even though $A$ is only a subset of $V$. Moreover, all such $D$ entries are determined for $A$. This is because (1) $A$ is initialized with this property (2) whenever a new vertex $v$ is about to be added into $A$, $D[u_i,v]=D[v,u_i]$ is calculated for every $u_i\in A$. Therefore, as $A$ eventually expands to $V$, we get the full $D$ matrix.

Furthermore, the correctness of this algorithm relies on the correctness of the following equation
\begin{equation*}
D[u_i,v]=\left\{\begin{array}{ll}D[u_i,v.father]&\textup{if }w(D[u_i,v.father])\geqslant w(v,v.father)\\(v,v.father)&\textup{otherwise}\end{array}\right.
\end{equation*}

Since the new vertex $v$ connects to $A$ at $v.father\in A$, the unique path from $v$ to $u_i\in A$ is the union of $(v,v.father)$ and the unique path from $v.father$ to $u_i$. Therefore, the heaviest edge must be either $(v,v.father)$ or $D[u_i,v.father]$, depending on which has a higher weight. This proves the correctness of the equation above.

~

\noindent\textbf{Runtime}:

The algorithm visits $V$ vertices. At vertex $v$, it calculates $D[u,v]$ for all $u\in A$. For one $v$, this means $O(V)$ operations, each of which contains constant times of comparing and value assigning. Therefore, the total runtime is $O(V)\times O(V)=O(V^2)$.

\subsection{Part (c)}
\noindent\textbf{Description}:

For each $(u,v)\in E$, consider the tree $S_{u,v}=T\cup\{(u,v)\}-\{D[u,v]\}$. The $S_{u,v}$ with minimal weight is the second-best spanning tree. See the pseudocode.
\begin{algorithm}
\caption{Finding the second-best spanning tree}
\begin{algorithmic}[1]
\Procedure{SecondBest}{$T,E,V$}
\State{$D=$\Call{LongestEdge}{$T$}}
\State{$maxWeightReduce\gets-\infty$}
\For{$(u,v)$ in $E$}
	\If{$w(D[u,v])-w(u,v)>maxWeightReduce$}
		\State{$maxWeightReduce\gets w(D[u,v])-w(u,v)$}
		\State{$(u^*,v^*)\gets(u,v)$}
	\EndIf
\EndFor
\State{$S\gets T\cup\{(u^*,v^*)\}-\{D[u^*,v^*]\}$}
\State{\Return{S}}
\EndProcedure
\end{algorithmic}
\end{algorithm}

~

\noindent\textbf{Correctness}:

According to part (a), $S$ only differs from $T$ by one edge. Suppose $(u,v)\in S$ but $(u,v)\notin T$. Then in order to avoid a loop, one of the edges along the original path connecting $u$ and $v$ must be removed from $S$. In order to minimize $w(S)$, the optimal choice of this edge is the heaviest one, $D[u,v]$. By looping over all possible $(u,v)\in E$, we compare across different $S_{u,v}=T\cup\{(u,v)\}-\{D[u,v]\}$. The lightest of all is the lightest possible spanning tree that differs from $T$ by exactly one edge. Part (a) tells us that it must be the second-best spanning tree $S$.

~

\noindent\textbf{Runtime}:

\textsc{LongestEdge} costs $O(V^2)$. Then the algorithm loops through all $|E|$ vertex pairs. In each iteration only constant amount of work is done. Therefore, the loops runs at $O(E)=O(V^2)$ time. Altogether, this algorithm has a runtime of $O(V^2)$.

\end{document}
