\documentclass{article}
\usepackage[utf8]{inputenc}
\usepackage{amsmath}
\usepackage{amssymb}
\usepackage{indentfirst}
\usepackage{color}
\usepackage{forest}
\usepackage{fancyhdr}
\usepackage{algorithm}
\usepackage[noend]{algpseudocode}
\usepackage{geometry}
\geometry{left=2.5cm,right=2.5cm,top=2.5cm,bottom=2.5cm}

\title{6.046/18.410 Problem Set 6}
\author{Yijun Jiang\vspace{3pt}\\Collaborator: Hengyun Zhou, Eric Lau}
%\email{yjjiang@mit.edu}
\date{\today}

\pagestyle{fancy}
\lhead{Yijun Jiang}
\rhead{6.046/18.410 Problem Set 6}

\begin{document}
\maketitle
\section{Shipping to Save the World}
\subsection{Part (a)}
\subsubsection{Subpart (i)}
\noindent\textbf{Description}:

Create a tripartite graph $G=(V,E)$, where $V=D\cup S\cup C$. $D$ is the set of day vertices, $S$ the set of ship vertices and $C$ the set of city vertices. $|D|=7,|S|=m$ and $|C|=n$. If on day $d_i$ a ship $s_j$ can operate, create a directed edge $(d_i,s_j)$. If a ship $s_j$ can reach the city $c_k$, create a directed edge $(s_j,c_k)$.

Construct a flow network by adding a source $s$ and a sink $t$. For every $d_i\in D$, create a directed edge $(s,d_i)$ with capacity $c(s,d_i)=v_i$. For every $c_k\in C$, create a directed edge $(c_k,t)$ with capacity $c(c_k,t)=f_k$. Let $V'$ and $E'$ be the new sets of vertices and edges, and $G'=(V',E')$. All edges between $D$ and $S$ as well as between $S$ and $C$ have capacity $\infty$.

Run a maximum flow algorithm (e.g. Edmonds-Karp) on $G'$. If $|f|=\sum_kf_k$, a shipping plan exists such that each city $c_k$ receives exactly $f_k$ supplies.

~

\noindent\textbf{Correctness}:

A flow in $G'$ is equivalent to a shipping plan subject to the constraints that (a) each ship can operate on certain days and can reach certain cities (b) on day $d_i$ no more than $v_i$ supplies are sent (c) for city $c_k$ NO MORE THAN $f_i$ supplies are received. Among these constraints, (a) is enforced by the structure of $G'$, while (b) and (c) are enforced by the finite capacities of $(s,d_i)$ and $(c_k,t)$, respectively. If a flow exists such that $f(c_k,t)=c(c_k,t)=f_k$ for all $c_k$, each city $c_k$ gets exactly $f_k$ supplies. This requirement is equivalent to $|f|=\sum_kf_k$. Such a flow must be a maximum flow because it saturates the cut $(V'/\{t\})$-$\{t\}$. Therefore, to determine the existence of a shipping plan, where the constraint (c) is strengthened to ``for city $c_k$ EXACTLY $f_i$ supplies are received", we can solve the maximum flow problem of $G'$ and see if $|f|=\sum_kf_k$. If so, such a shipping plan exists and can be determined by $f$ (see the following subpart). Otherwise, such a shipping plan does not exist. This proves the correctness of the algorithm.

~

\noindent\textbf{Runtime}:

$|V'|=O(|D|+|S|+|C|)=O(m+n)$. Notice that $|D|=7$ and each $s_j$ can only reach $|Y_j|\leqslant5$ cities. Therefore, $|E'|=O(7|S|+5|S|+|D|+|C|)=O(m+n)$. Construction of $G'$ costs $O(V'+E')=O(m+n)$ time. Solving the maximum flow problem costs $O(V'E'^2)=O((m+n)^3)$ by Edmonds-Karp, or $O(V'^3)=O((m+n)^3)$ by relabel-to-front. Finally, checking if $|f|=\sum_kf_k$ costs $O(n)$. Therefore, the total runtime is limited by the maximum flow solver. Since $m<n$, the runtime is simplified as $T=O(n^3)$.

\subsubsection{Subpart (ii)}
\noindent\textbf{Description}:

The algorithm described above (named \textsc{IfPlanExists}) not only determines whether a valid shipping plan exists, but also returns the maximum flow $f$ in the graph $G'$. Suppose a valid shipping plan
exists. On day $d_i$, ship $s_j$ delivers in total $f(d_i,s_j)$ supplies. We determine this plan by specifying how the $f(d_i,s_j)$ supplies are distributed to $c_k$.

For a given $s_j$, construct a flow network $G_j=(V_j,E_j)$ as the following. Create 7 vertices corresponding to $d_i$, and $|Y_j|\leqslant5$ vertices corresponding to all the cities $c_{k_\alpha}$ that are reachable from $s_j$, where $\alpha=1,2,\cdots,|Y_j|$. For every $i$ and every $\alpha$, create a directed edge $(d_i,c_{k_\alpha})$ of capacity $\infty$. We also introduce a source $s$ and a sink $t$. For every $i$, create a directed edge $(s,d_i)$ whose capacity is $c^j(s,d_i)=f(d_i,s_j)$. For every $\alpha$, create a directed edge $(c_{k_\alpha},t)$ whose capacity is $c^j(c_{k_\alpha},t)=f(s_j,c_{k_\alpha})$. Solve the maximum flow problem in this network. Let the resulting flow be $f^j$. Then $f^j(d_i,c_{k_\alpha})$ gives the amount of supplies delivered on day $d_i$ by ship $s_j$ to city $c_{k_\alpha}$.

Iterating over all $s_j$ produces a valid shipping plan.

%We keep a list of length $n$ storing the shortage of supplies in each city. Call this list $A$. Initially set $A[k]=f_k$ for all $k$. Also, for each edge $(d_i,s_j)$, keep a list $B_{ij}$ of length $n$ such that $B_{ij}[k]$ stores the amount of supplies that are delivered to city $c_k$ by ship $s_j$ on day $d_i$. $B_{ij}[k]$ can be conveniently stored in a 3D matrix of size $7\times m\times n$. Initially set $B_{ij}[k]=0$ for all $i,j,k$.

%Loop over $D=\{d_i\}$. Run an inner loop over $S=\{s_j\}$. Inside this inner loop where $d_i$ and $s_j$ are fixed, use a variable $x$ to keep tract of how many supplies are still available for this ship on this day. Initially set $x=f(d_i,s_j)$. Run an innermost loop over $C=\{c_k\}$. For each $k$, set $B_{ij}[k]=\min(A[k],x)$ and update $A[k]$ as well as $x$ by subtracting $B_{ij}[k]$ from them. For efficiency, we do not need to loop over all $k$: once $x$ is exhausted, we can stop iterating over $k$.

%When the outer loop goes through all 7 days, the shipping plan is determined. Namely, $B_{ij}[k]$ specifies on day $d_i$, how many supplies does ship $s_j$ deliver to city $c_k$.

~

\noindent\textbf{Correctness}:

For the flow $f$ determined by \textsc{IfPlanExists}, $\sum_if(d_i,s_j)=\sum_kf(s_j,c_k)=\sum_\alpha f(s_j,c_{k_\alpha})$ due to flux conservation at $s_j$.

In the flow network $G_j$ corresponding to $s_j$, no capacity constraint is exerted on any $(d_i,c_{k_\alpha})$. Therefore, the only cuts with finite capacities are $\{s\}$-$(V_j/\{s\})$ and $(V_j/\{t\})$-$\{t\}$. The former has capacity $\sum_if(d_i,s_j)$ and the latter has capacity $\sum_\alpha f(s_j,c_{k_\alpha})$. These two values are equal and thus determine the minimal cut capacity, which is also the maximum flow. Therefore, all edges coming out of $s$ and all edges going into $t$ are saturated.

Because of this, $\sum_\alpha f^j(d_i,c_{k_\alpha})=f^j(s,d_i)=c^j(s,d_i)=f(d_i,s_j)$ and $\sum_if^j(d_i,c_{k_\alpha})=f^j(c_{k_\alpha},t)=c^j(c_{k_\alpha},t)=f(s_j,c_{k_\alpha})$. We can then think of $f^j(d_i,c_{k_\alpha})$ as the flow in the original network going from $d_i$ to $c_{k_\alpha}$ via $s_j$. So $f^j(d_i,c_{k_\alpha})$ gives the amount of supplies delivered on day $d_i$ by ship $s_j$ to city $c_{k_\alpha}$. It follows that iterating over all $s_j$ produces a valid shipping plan.

~

\noindent\textbf{Runtime}:

Suppose \textsc{IfPlanExists} returns a valid $f$. Notice that for every $s_j$, the flow network $G_j$ is of constant size: at most $7+5+2=14$ vertices. Therefore, solving the maximum flow problem in $G_j$ costs $O(1)$ time. Since we iterate over all $s_j$, the total runtime is $T=O(m)$.

We see that determining a shipping plan is of linear time once \textsc{IfPlanExists} is already executed. If not, or if somehow $f$ is discarded, we have to run \textsc{IfPlanExists} first and altogher $O(n^3)$ time is required.

\subsection{Part (b)}
\noindent\textbf{Description}:

WLOG, let $s_1$ be the ship that cannot be operated. We know the previously valid maximum flow $f_{old}$. Now create a new flow in the network without $s_1$ as well as edges to and from $s_1$. Call this network $G_{new}$. Let $f_{new}(s,d_i)=f_{old}(s,d_i)-f_{old}(d_i,s_1)$ for all $i$ and $f_{new}(c_k,t)=f_{old}(c_k,t)-f_{old}(s_1,c_k)$ for all $k$ (it is sufficient to consider only $c_k$ reachable from $s_1$). And $f_{new}(e)=f_{old}(e)$ if $e$ goes to or from some $s_i$ other than $s_1$.

It is proved in the correctness part that $G_{new}$ is a flow network and $f_{new}$ is a flow in $G_{new}$. But $f_{new}$ may not be the maximum flow. So we augment it by Ford-Fulkerson method. If at the end of augmentation $|f|=\sum_kf_k$, then it is possible to make a shipping plan without $s_1$. Otherwise, such a plan is impossible.

~

\noindent\textbf{Correctness}:

The new network $G_{new}$ is a flow network because we only delete $s_1$ as well as all $(d_i,s_1)$ and $(s_1,c_k)$. No backward edges are introduced.

$f_{new}$ still satisfies flux conservation at all $d_i$, because $f_{new}(s,d_i)=f_{old}(s,d_i)-f_{old}(d_i,s_1)=\sum_jf_{old}(d_i,s_j)-f_{old}(d_i,s_1)=\sum_{j\neq1}f_{old}(d_i,s_j)=\sum_{j\neq1}f_{new}(d_i,s_j)$. Similarly, flux conservation holds at all $c_k$. Flux conservation at $s_j$ other than $s_1$ is also maintained because all flows into and out of such an $s_j$ are not changed. Finally, capacity constraints are satisfied because $f_{new}(e)\leqslant f_{old}(e)$ for all $e$. Moreover, for those $f_{new}(e)$ that are reduced, they are still non-negative. This is because $f_{new}(s,d_i)=\sum_{j\neq1}f_{old}(d_i,s_j)\geqslant0$, and the same is true for $f_{new}(c_k,t)$. In conclusion, $f_{new}$ is a valid network flow on $G_{new}$.

Therefore, we can apply Ford-Fulkerson method to augment $f_{new}$. At the end we get a maximum flow $f$. If $|f|=\sum_kf_k$, then from the previous analysis, the demands of all cities can be satisfied. So a valid shipping plan still exists. Otherwise, we cannot find a valid shipping plan.

~

\noindent\textbf{Runtime}:

To avoid confusion, I use capital $K$ to replace the lowercase $k$ in the problem set. Suppose $s_1$ delivered in total $K$ supplies. Then the loss of $|f_{new}|$ is $\sum_i(f_{old}(s,d_i)-f_{new}(s,d_i))=\sum_if_{old}(d_i,s_1)=K$. Therefore, $|f_{new}|=\sum_kf_k-K$. By augmentation, we can at most improve the value of flow by $K$. Since each iteration of Ford-Fulkerson augments the flow by at least 1, $O(K)$ iterations are required. Each runs in $O(E)=O(m+n)=O(n)$ time. Therefore, the algorithm costs $T(K)=O(nK)$ time in all.


\section{Career Fair}
\noindent\textbf{Description}:

We create a bipartite graph $G=(V,E)$, where $V=L\cup R$. $L$ is the set of student vertices and $R$ the set of employer vertices. $|L|=n$ and $|R|=m$. For every $l_i\in L$ and every $r_j\in R$, create a directed edge $(l_i,r_j)$. There are in total $mn$ edges.

Construct a flow network by adding a source $s$ and a sink $t$. For every $l_i\in L$, create a directed edge $(s,l_i)$. For every $r_j\in R$, create a directed edge $(r_j,t)$. Let $V'$ and $E'$ be the new sets of vertices and edges, and $G'=(V',E')$. Let the capacity of every edge in $E'$ be 1.

Sort $t_{i,j}$ (break ties arbitrarily) and store them in an array $T$. Recursively reduce the size of $T$ by bisection to get rid of large $t_{i,j}$ until a student does not have any employer to talk to. We need to keep tract of the ``active subarray" of $T$ that is being bisected. Call this subarray $T'$. In reality we just store the start point and the end point of $T'$. Initially, set $T'=T$. If $n>m$, raise an exception saying that some students are unable to talk to any employer. Otherwise, do the following recursion.

Bisect $T'=T_1'\cup T_2'$, where $T_1'$ contains the smaller half of $T'$. For all the edges whose $t_{i,j}$ are stored in $T_2'$, set their capacities to 0, thus killing them in the flow network. Find a maximum bipartite matching of $G$ by solving the maximum flow problem in $G'$. If $|M|=n$, let $T'=T_1'$ and recurse. If $|M|<n$, for all the edges whose $t_{i,j}$ are stored in $T_2'$, set their capacities back to 1. Let $T'=T_2'$ and recurse. $|M|>n$ is impossible since the cut $\{s\}$-$(V'/\{s\})$ has capacity $n$.

The recursion goes on until $T'$ contains only one entry $t_{i_0,j_0}$. Run the maximum flow solver on $G'$ (where the capacities of all edges beyond $(i_0,j_0)$ are already set to 0). The maximum bipartite matching assigns students to employers such that each student has an employer to speak with and the last student's travel time is minimized to $t_{i_0,j_0}$.

~

\noindent\textbf{Correctness}:

Matching students with employers such that any two students do not go to the same employer and any student does not go to two different employers is a maximum-bipartite-matching problem between $L$ and $R$. This can be solved as a network flow problem by adding a source and a sink and setting all edge capacities to be 1. The rule that each student must speak with exactly one employer further requires that every vertex in $L$ is matched up with a vertex in $R$. Therefore, the maximum bipartite matching, i.e. the value of the maximum flow, must be exactly $n$. This can be achieved when $n\leqslant m$, which is checked at the beginning.

Suppose $n\leqslant m$ and let the minimized maximum of $t_{i,j}$ in a solution be $t^*=t_{i*,j*}$. The corresponding network flow forbids any edge $(i,j)$ whose $t_{i,j}>t^*$. Moreover, if we further forbid $(i^*,j^*)$ from carrying a flow, we do not get a solution with a smaller time because $t^*$ is already optimal. This means that we either get the same $t^*$ (which may happen if there are multiple edges with this time cost), or we cannot find any maximum bipartite matching such that $|M|=n$. In the first case, we can always keep on killing the most time-consuming edge until $|M|=n$ does not hold (which must happen no later than all edges of $t^*$ being killed). Therefore, a valid strategy is to sort $t_{i,j}$, break ties arbitrarily, and find the frontmost one (call it $t^{**}$) beyond which all edges can be killed without affecting $|M|=n$. This is achieved by bisection in the algorithm.

In the bisection process it is guaranteed that $t^{**}\in T'$. This claim is proved by induction as the following. At the beginning the claim holds because $T'=T$. Suppose $t^{**}\in T'$ in the $i$-th recursion. In the $(i+1)$-th recursion, if killing all edges in $T_2'$ preserves $|M|=n$, it must be that $t^{**}\in T_1'$. Otherwise $t^{**}\in T_2'$. By setting $T'=T_1'$ or $T_2'$ respectively, it is guaranteed that $t^{**}\in T'$ in this recursion. This means $t^{**}\in T'$ always holds. After finite recursions, $T'$ will be reduced to one element $t_{i_0,j_0}$. Then $t^{**}=t_{i_0,j_0}$ is the minimized maximum of travel time. The corresponding maximum bipartite matching $M$ gives a valid assignment of students to employers.

~

\noindent\textbf{Runtime}:

$|V'|=O(m+n)$ and $|E'|=O(mn)$. Construction of $G'$ costs $O(V'+E')=O(mn)$. $|T|=mn$, so bisection requires $O(\log(mn))$ recursions. In each recursion, reseting capacities costs $O(mn)$ time, and solving a maximum-bipartite matching problem by a network flow algorithm (e.g. Ford-Fulkerson) costs $O(V'E')=O(mn(m+n))$ time. Other work (e.g. comparisons) costs constant time. In conclusion, the total time cost is $T=O(mn(m+n)\log(mn))$. Notice that $n\leqslant m$ is required for a solution to exist. This simplifies the time cost to $T=O(m^2n\log m)$.

\end{document}
