\documentclass{article}
\usepackage[utf8]{inputenc}
\usepackage{amsmath}
\usepackage{amssymb}
\usepackage{indentfirst}
%\usepackage{forest}
%\usepackage{listings}
\usepackage{algorithm}
\usepackage[noend]{algpseudocode}
\usepackage{geometry}
\geometry{left=2.5cm,right=2.5cm,top=2.5cm,bottom=2.5cm}
\makeatletter
\def\BState{\State\hskip-\ALG@thistlm}
\makeatother

\title{6.046/18.410 Problem Set 2}
\author{Yijun Jiang\\Collaborator: Eric Lau, Hengyun Zhou}
%\email{yjjiang@mit.edu}
\date{\today}

\begin{document}
\maketitle

\section{Finding the $k$-th smallest element}
\subsection{Part (a)}
\noindent\underline{Description}: We can apply merge sort since both $A$ and $B$ are sorted, and stop once we get the $k$-th element of the merged array. This element is the $k$-th smallest element in the union of $A$ and $B$. More efficiently, we do not need to store the merged array.

\noindent\underline{Correctness}: The correctness of this algorithm is guaranteed by the correctness of merge sort. In merge sort, the $k$-th element being added to the merged array is the $k$-th smallest element in the union of two original arrays.

\noindent\underline{Runtime}: In this algorithm, we need to index into an array $k+1$ times, and make $k$ comparisons between two elements. Since finding an element by its index and comparing two elements are $O(1)$ operations, the runtime of this algorithm is $O(k)$.

\subsection{Part (b)}
\noindent\underline{Description}: We can compare the $\lfloor\frac{k}{2}\rfloor$-th element of both $A$ and $B$. If $A[\lfloor\frac{k}{2}\rfloor]\leqslant B[\lfloor\frac{k}{2}\rfloor]$, then $A[1],\cdots,A[\lfloor\frac{k}{2}\rfloor]$ are guaranteed to be no greater than the $k$-th smallest in the union of $A$ and $B$, which for convenience will be referred to as $x$. These $\lfloor\frac{k}{2}\rfloor$ elements are identified. Otherwise, $B[1],\cdots,B[\lfloor\frac{k}{2}\rfloor]$ are no greater than $x$ and are identified.

\noindent\underline{Correctness}: Suppose $A[\lfloor\frac{k}{2}\rfloor]\leqslant B[\lfloor\frac{k}{2}\rfloor]$. Let $i_A$ and $i_B$ be the largest indices in $A$ and $B$ such that $A[i_A]\leqslant x$ and $B[i_B]\leqslant x$. Notice that $A$ and $B$ are sorted, so the ordering of indices is equivalent to the ordering of elements in either array.

If $i_A<\lfloor\frac{k}{2}\rfloor$, then according to the definition of $i_A$, $x<A[\lfloor\frac{k}{2}\rfloor]$. Therefore, $x<B[\lfloor\frac{k}{2}\rfloor]$ and thus $i_B<\lfloor\frac{k}{2}\rfloor$. As a result, in the union of $A$ and $B$, there are only $i_A+i_B<k$ elements that are no greater than $x$, which contradicts the fact that $x$ is the $k$-th smallest element in the union of $A$ and $B$. Therefore, it must be that $i_A\geqslant\lfloor\frac{k}{2}\rfloor$, which implies that $A[1],\cdots,A[\lfloor\frac{k}{2}\rfloor]\leqslant A[i_A]\leqslant x$. The proof is similar if $A[\lfloor\frac{k}{2}\rfloor]>B[\lfloor\frac{k}{2}\rfloor]$. Then we have $B[1],\cdots,B[\lfloor\frac{k}{2}\rfloor]\leqslant B[i_B]\leqslant x$.

\noindent\underline{Runtime}: To identify $\lfloor\frac{k}{2}\rfloor$ of the $k$ smallest elements, we need to indice once into $A$ and once into $B$, and make one comparison. Altogether this takes $O(1)$ time.

\subsection{Part (c)}
\noindent\underline{Description}: Once we identify half of the elements that are no greather than the $k$-th smallest element $x$, the rank of $x$ in the union of unidentified subarrays is halved. We can recursively cut down this rank by half. The pseudocode below shows the details.
\begin{algorithm}
\caption{Find the $k$-th minimum of $A\cup B$ in $O(\log k)$ time}
\begin{algorithmic}[1]
\Procedure{RecursiveFind}{$A,B,k$}
\If{k=1}
	\Return{$\min(A[1],B[1])$}
\Else
	\If{$A[\lfloor\frac{k}{2}\rfloor]\leqslant B[\lfloor\frac{k}{2}\rfloor]$}
		\Return{\Call{RecursiveFind}{$A[\lfloor\frac{k}{2}\rfloor+1:\textup{end}],B,\lceil\frac{k}{2}\rceil$}}
	\Else{}
		\Return{\Call{RecursiveFind}{$A,B[\lfloor\frac{k}{2}\rfloor+1:\textup{end}],\lceil\frac{k}{2}\rceil$}}
	\EndIf
\EndIf
\EndProcedure
\end{algorithmic}
\end{algorithm}

\noindent\underline{Correctness}: Still let $x$ be the element we look for, and let $\textup{rank}(x)$ be the rank in the union of unidentified subarrays. What we do is we identify and throw away $\lfloor\frac{\textup{rank}(x)}{2}\rfloor$ elements in each iteration. Since these elements are no greater than $x$, it is guaranteed that $\textup{rank}(x)$ is halved. The base case is when $\textup{rank}(x)=1$, where taking $\min$ of the first entries of both unidentified subarrays give the $x$ we want.

\noindent\underline{Runtime}: The runtime of this algorithm is given by the recursion
\begin{equation*}
T(k)=T(k/2)+O(1)
\end{equation*}
whose solution, by the master theorem, is $T(k)=O(\log k)$.

\section{Structures related to van Emde Boas}
All the tries in this problem are considered to be nonempty. Otherwise we should raise an exception for relevant operations.

\subsection{Part (a)}
\noindent\underline{Description}: There are two main cases: (1) $x$ exists (2) $x$ does not exist in the trie. The following algorithm deals with both cases.

We first go down the tree trying to reach $x$. This is done by going deeper into the levels to match the prefix of $x$ (or equivalently, we call \textsc{Find}). If $x$ exists, we reach a leaf. Otherwise, we come to a node where any deeper leaf/node does not match (the prefix of) $x$. We then go up from this leaf/node until, either for the first time we reach a node who has an unexplored branch on the right (which means we come from the left), or this never happens, in which case we raise a no-successor exception. If we find the node, go all the way down its right branch, making sure that we take the left branch whenever possible. The leaf $y$ that we eventually reach is the successor of $x$. The pseudocode below makes it clearer.
\begin{algorithm}
\caption{Find the successor of $x$ in a trie (whose root is $r$) in $O(\log U)$ time}
\begin{algorithmic}[1]
\Procedure{Successor}{$x,r$}
\State{$px$=\textsc{Find}$(x,r)$}
\State\Comment{If $x$ does not exist, suppose \textsc{Find} returns the deepest node that matches the prefix of $x$.}
\State{Go up from $px$ until either (1) reaching the first node $n$, where $px$ is in left branch of $n$, and right branch of $n$ is non-empty (2) reaching $r$, but $r$ does not satisfy (1)}
\If{Case (1)}
	\State{$y$=\textsc{MinKey}($n$)}\Comment{Find the minimum key in the sub-trie whose root is $n$}
\Else{}
	\State{No-successor exception: $x$ is (or is greater than if $x\notin$ trie) the maximum key in the trie}
\EndIf
\EndProcedure
\end{algorithmic}
\end{algorithm}

\noindent\underline{Correctness}: The correctness of \textsc{Successor} is guaranteed by the trie structure: the successor of $x$ is always ``the leftmost leaf in the closest right subtrie".

\noindent\underline{Runtime}: There are three main steps: reaching $x$, going up, reaching $y$. Each step visits $O(\log U)$ levels, thus $T_{\textsc{Successor}}=O(\log U)$.

\subsection{Part (b)}
\textsc{Insert} still takes $O(\log U)$ time. Think of the worst case where the root has only one branch and we insert a key into the other (originally empty) branch. Then we have to create $\log U$ nodes (including a leaf). Generally we need to create $O(\log U)$ nodes and a leaf, each of which is $O(1)$ time. So $T_{\textsc{Insert}}=O(\log U)$.

\textsc{Find} takes $O(1)$ time, since we can query $H_{leaf}(x)$ in $O(1)$ (say, by hashing into a leaf).

\textsc{Remove} still takes $O(\log U)$ time. We still need to remove the leaf and nodes bottomup, until reaching a node that is nonempty. Therefore, $O(\log U)$ removals must be done, each of which is $O(1)$ time. So $T_{\textsc{Remove}}=O(\log U)$.

\subsection{Part (c)}
\noindent\underline{Description}: We need to bisect the levels (or equivalently, bisect the prefix of $x$) in order to reach $O(\log\log U)$ runtime. The goal is to find a node $n$ whose $\max$ is $x$ (or predecessor of $x$ if $x$ is not in the trie), but whose father's $\max$ is greater than $x$. Then this father node must have a second (and right) child, whose $\min$ gives the $y$ we want. The pseudocode is shown below. Still, it can deal with both cases: $x$ exists and $x$ does not exist in the trie.
\begin{algorithm}
\caption{Find the successor of $x$ in an augmented trie (whose root is $r$) in $O(\log\log U)$ time}
\begin{algorithmic}[1]
\Procedure{Successor}{$x,r$}
\If{$x\geqslant r.max$}
	\State{No-successor exception: $x$ is (or is greater than if $x\notin$ trie) the maximum key in the trie}
\ElsIf{$x<r.min$}
	\Return{$r.min$}
	\State\Comment{These two edge cases cannot he handled by the pseudocode below}
\Else{}
	\State{$n1\gets$\Call{Bisect}{$x,1,\log U$}}
	\State\Comment{See detailed pseudocode below for $\textsc{Bisect}$}
	\State{$n2=n1.father.rightChild$}
	\State{\Return{$n2.min$}}
\EndIf
\EndProcedure
\State{}
\Procedure{Bisect}{$x,h\_min,h\_max$}
\State\Comment{Return a leaf/node $n$ such that $n.max=x$ (or predecessor of $x$ if $x\notin$ trie) but $n.father.max>x$}
\State\Comment{$h$ is the level in the trie, or the length of prefix of $x$; note that $r$ is level $0$}
\If{$h\_min=h\_max$}
	\Return{\Call{Hash}{$x[1:h\_min]$}}
	\State\Comment{Except for the above edge cases, this \textsc{Hash} never returns NULL, see the Correctness part}
	\Else
	\State{$h\_mid\gets\lfloor\frac{1}{2}(h\_min+h\_max)\rfloor$}
	\State{$n\_mid\gets\textsc{Hash}(x[1:h\_mid])$}
	\If{$n\_mid=\textup{NULL}$ \textbf{or} $n\_mid.max=x$}
		\Return{\Call{Bisect}{$x,h\_min,h\_mid$}}
		\State\Comment{$n\_mid=\textup{NULL}$ means $x$ does not exist in the trie}
	\Else{}
		\Return{\Call{Bisect}{$x,h\_mid+1,h\_max$}}
	\EndIf
\EndIf
\EndProcedure
\end{algorithmic}
\end{algorithm}

\noindent\underline{Correctness}: Suppose $r.min\leqslant x<r.max$, otherwise it is an edge case treated separately in $O(1)$ time. First let us suppose $x$ is in the trie. Then it is guaranteed that there is a node $n$ such that $n.max=x$ but $n.father.max>x$. This is because, there is always a path connecting $r$ to the leaf $x$. Going down this path, $\max$ decreases monotonically from $r.max$ to $x$. The highest level where $\max=x$ gives the desired $n$. If $x$ does not exist in the trie, in the reasoning above replace $x$ by the predecessor of $x$, and the existence of $n$ can still be proved. Thus the base case of \textsc{Bisect} never returns NULL. Finally, $n$ is not $r$ since $r.max>x$, thus $n$ always has a father.

Instead of searching level-by-level, bisection search is more an efficient way to locate $n$. Bisection works since $\max$ is monotonically decreasing down this path.

Any leaf in the subtrie $n$ is no greater than $x$. But since $n.father.max>x$, we know that $n.father$ must have another branch containing a bigger $\max$. So this branch is to the right of $n$. This means every key in this branch is larger than $x$. Moreover, the $\min$ in this branch is the smallest key that is still larger than $x$. It must be the successor of $x$. Thus the correctness is proved. 

\noindent\underline{Runtime}: \textsc{Successor} calls \textsc{Bisect}, which recursively calls itself. The following steps in \textsc{Successor} takes constant time. So the entire runtime is limited by \textsc{Bisect}.

The original input size for \textsc{Bisect} is $h=\log U$. The input size is halved in each recursive call. The rest operations, including \textsc{Hash}, takes only $O(1)$ time. Then we have the following recursion
\begin{equation*}
T(h)=T(h/2)+O(1)
\end{equation*}
whose solution, by the master theorem, is $T(h)=O(\log h)=O(\log\log U)$. Thus, $T_{\textsc{Successor}}=O(\log\log U)$.

\end{document}
