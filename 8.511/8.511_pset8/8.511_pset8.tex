\documentclass{article}
\usepackage{amsmath}
\usepackage{amssymb}
\usepackage{indentfirst}
\usepackage{graphicx}
\usepackage{color}
\usepackage{fancyhdr}
\usepackage{epstopdf}
\usepackage{indentfirst}
\usepackage{geometry}
\usepackage{bm}
\geometry{left=2.5cm,right=2.5cm,top=2.5cm,bottom=2.5cm}

\title{8.511 Problem Set 8}
\author{Yijun Jiang}
%\email{yjjiang@mit.edu}
\date{\today}

\pagestyle{fancy}
\lhead{Yijun Jiang}
\rhead{8.511 Problem Set 8}

\begin{document}
\maketitle
\section{Hartree-Fock}
Define one-body Hamiltonian
\begin{equation*}
H_1=-\frac{\hbar^2}{2m}\nabla^2-\frac{Ze^2}{r}
\end{equation*}

I use $J_i$ instead of $V_H$ to denote the Hartree potential. Then the Hartree-Fock equation is written as
\begin{equation*}
H_1u_i(1)+J_iu_i(1)-\sum_{j\neq i,\textup{spin}\parallel}\int d\mathbf{r}_2u_j^*(2)\frac{e^2}{r_{12}}u_j(1)u_i(2)=\varepsilon_iu_i(1)
\end{equation*}

The parallel-spin constraint in the exchange summation can be removed, for the integral vanishes whenever $i$ and $j$ have anti-parallel spins.

The Hartree potential is
\begin{align*}
J_i&=\sum_{j\neq i}\int d\mathbf{r}_2u_j^*(2)\frac{e^2}{r_{12}}u_j(2)\\
&=\left(\sum_j\int d\mathbf{r}_2u_j^*(2)\frac{e^2}{r_{12}}u_j(2)\right)-\int d\mathbf{r}_2u_i^*(2)\frac{e^2}{r_{12}}u_i(2)\\
&=J'-\Delta J
\end{align*}
where
\begin{align*}
&J'=\sum_j\int d\mathbf{r}_2u_j^*(2)\frac{e^2}{r_{12}}u_j(2)\\
&\Delta J=\int d\mathbf{r}_2u_i^*(2)\frac{e^2}{r_{12}}u_i(2)
\end{align*}

The exchange term is (the spin constraint removed)
\begin{align*}
\hat{K_i}u_i(1)&=\sum_{j\neq i}\int d\mathbf{r}_2u_j^*(2)\frac{e^2}{r_{12}}u_j(1)u_i(2)\\
&=\left(\sum_j\int d\mathbf{r}_2u_j^*(2)\frac{e^2}{r_{12}}u_j(1)u_i(2)\right)-\int d\mathbf{r}_2u_i^*(2)\frac{e^2}{r_{12}}u_i(1)u_i(2)\\
&=\hat{K'}u_i(1)-\Delta\hat{K}u_i(1)
\end{align*}
where
\begin{align*}
&\hat{K'}u_i(1)=\sum_j\int d\mathbf{r}_2u_j^*(2)\frac{e^2}{r_{12}}u_j(1)u_i(2)\\
&\Delta\hat{K}u_i(1)=\int d\mathbf{r}_2u_i^*(2)\frac{e^2}{r_{12}}u_i(1)u_i(2)
\end{align*}

Notice that
\begin{equation*}
\Delta Ju_i(1)=\Delta\hat{K}u_i(1)
\end{equation*}

Therefore, the Hartree-Fock equation can be written as
\begin{equation*}
H_1u_i(1)+J'u_i(1)-\hat{K'}u_i(1)=\varepsilon_iu_i(1)
\end{equation*}

Taking dot product with $u_k(1)$, we get
\begin{equation}
\int d\mathbf{r}_1u^*_k(1)(H_1+J')u_i(1)-\sum_j\iint d\mathbf{r}_1d\mathbf{r}_2u_k^*(1)u_j^*(2)\frac{e^2}{r_{12}}u_j(1)u_i(2)=\varepsilon_i\int d\mathbf{r}_1u^*_k(1)u_i(1)\label{eq}
\end{equation}

Exchange $i$ and $k$, then take Hermitian conjugate. Notice that $H_1$ and $J'$ are both Hermitian. Therefore,
\begin{equation}
\int d\mathbf{r}_1u^*_k(1)(H_1+J')u_i(1)-\sum_j\iint d\mathbf{r}_1d\mathbf{r}_2u_k^*(2)u_j^*(1)\frac{e^2}{r_{12}}u_j(2)u_i(1)=\varepsilon^*_k\int d\mathbf{r}_1u^*_k(1)u_i(1)\label{hceq}
\end{equation}

Subtracting equation (\ref{hceq}) from equation (\ref{eq}), we have
\begin{equation*}
\sum_j\iint d\mathbf{r}_1d\mathbf{r}_2u_k^*(2)u_j^*(1)\frac{e^2}{r_{12}}u_j(2)u_i(1)-\sum_j\iint d\mathbf{r}_1d\mathbf{r}_2u_k^*(1)u_j^*(2)\frac{e^2}{r_{12}}u_j(1)u_i(2)=(\varepsilon_i-\varepsilon^*_k)\int d\mathbf{r}_1u^*_k(1)u_i(1)
\end{equation*}

The difference between the first term and the second term is a swap of arguments. Since $r_{12}=r_{21}$, we can exchange 1 and 2 in the first term, making it identical to the second. Therefore, the left hand side vanishes.
\begin{equation*}
(\varepsilon_i-\varepsilon^*_k)\int d\mathbf{r}_1u^*_k(1)u_i(1)=0
\end{equation*}

Let $i=k$, then $\varepsilon_i-\varepsilon^*_i=0$, which means $\varepsilon_i$ is real.

Let $i\neq k$. If no degeneracy occurs, $\varepsilon_i\neq\varepsilon_k$, so $\langle u_i|u_k\rangle=0$. If $\varepsilon_i$ and $\varepsilon_k$ are degenerate, there is some unitary matrix $\mathbf{T}$ that diagonalizes the degenerate subspace. Later I will prove that this unitary transformation preserves the Hartree-Fock equation. Then there is a way to choose $u_i$ and $u_k$ such that $\langle u_i|u_k\rangle=0$ even if degeneracy occurs.

Therefore, the solutions are orthonormal: $\langle u_i|u_j\rangle=\delta_{ij}$

~

\noindent\underline{Proof of the fact that $\mathbf{T}$ preserves the Hartree-Fock equation}:

Let $\tilde{\mathbf{u}}=\mathbf{T}\mathbf{u}$, where $\mathbf{u}=(u_1,u_2,\cdots,u_n)^T$ are the original states, and $\tilde{\mathbf{u}}=(\tilde{u}_1,\tilde{u}_2,\cdots,\tilde{u}_n)^T$ are the diagonalized states. Since $H_1$ is linear, $H_1(\mathbf{T}\mathbf{u}(1))=\mathbf{T}H_1\mathbf{u}(1)$.

The Hartree potential is unaffected by the uniform transformation, i.e. $\tilde{J'}(\mathbf{T}\mathbf{u}(1))=\mathbf{T}J'\mathbf{u}(1)$, for
\begin{align*}
\tilde{J'}(\mathbf{T}\mathbf{u}(1))&=\sum_j\int d\mathbf{r}_2\tilde{u}_j^*(2)\frac{e^2}{r_{12}}\tilde{u}_j(2)\tilde{\mathbf{u}}(1)\\
&=\int d\mathbf{r}_2\frac{e^2}{r_{12}}(\mathbf{u}^\dag(2)\mathbf{T}^\dag)(\mathbf{T}\mathbf{u}(2))(\mathbf{T}\mathbf{u}(1))\\
&=\int d\mathbf{r}_2\frac{e^2}{r_{12}}\mathbf{u}^\dag(2)\mathbf{u}(2)(\mathbf{T}\mathbf{u}(1))\\
&=\mathbf{T}\sum_j\int d\mathbf{r}_2u_j^*(2)\frac{e^2}{r_{12}}u_j(2)\mathbf{u}(1)\\
&=\mathbf{T}J'\mathbf{u}(1)
\end{align*}

The exchange operator is also unaffected by the uniform transformation, i.e. $\hat{\tilde{K'}}(\mathbf{T}\mathbf{u}(1))=\mathbf{T}\hat{K'}\mathbf{u}(1)$, for
\begin{align*}
\hat{\tilde{K'}}(\mathbf{T}\mathbf{u}(1))&=\sum_j\int d\mathbf{r}_2\tilde{u}_j^*(2)\frac{e^2}{r_{12}}\tilde{u}_j(1)\tilde{\mathbf{u}}(2)\\
&=\int d\mathbf{r}_2\frac{e^2}{r_{12}}(\mathbf{u}^\dag(2)\mathbf{T}^\dag)(\mathbf{T}\mathbf{u}(1))(\mathbf{T}\mathbf{u}(2))\\
&=\int d\mathbf{r}_2\frac{e^2}{r_{12}}\mathbf{u}^\dag(2)\mathbf{u}(1)(\mathbf{T}\mathbf{u}(2))\\
&=\mathbf{T}\sum_j\int d\mathbf{r}_2u_j^*(2)\frac{e^2}{r_{12}}u_j(1)\mathbf{u}(2)\\
&=\mathbf{T}\hat{K'}\mathbf{u}(1)
\end{align*}

Therefore, the uniform transformation preserves the Hartree-Fock equation.

\section{Schottky Barrier}
\subsection{Part (a)}
Before contact, the conduction band $E_c$ of the semiconductor is higher than the chemical potential $\mu$ of the metal, and the valance band $E_v$ is lower than $\mu$. The chemical potential of the n-type semiconductor is close to $E_c$ and thus higher than $\mu$. In contact and in equilibrium, the semiconductor chemical potential equals $\mu$, since the electric potential in the metal-semiconductor junction bends $E_c$ and $E_v$ downwards. Close to $z=0$, in the semiconductor both $E_c$ and $E_d$ are way higher than $\mu$ relative to $k_BT$, which means almost no population of electrons. This leads to a depletion region.

In the depletion region, the charge is dominated by ionized doner cores, whose density is $N_d$. The ionized electrons move to the metal side. In order to charge neutralize metal-semiconductor junction, these electrons will form a thin negatively charged layer on the metal boundary.

The boundary conditions are $\phi'(z_0)=0$ and $\phi(z_0)-\phi(0)=E_B/e$. Define the zero point of $\phi$ at $z=0$. Then the solution to Poisson's equation has the form
\begin{equation*}
\phi(z)=-\frac{2\pi N_de}{\varepsilon}((z-z_0)^2-z_0^2)
\end{equation*}
where
\begin{equation*}
\phi(z_0)=\frac{2\pi N_de}{\varepsilon}z_0^2=\frac{E_B}{e}
\end{equation*}

Therefore,
\begin{equation*}
z_0=\frac{1}{e}\sqrt{\frac{\varepsilon E_B}{2\pi N_d}}
\end{equation*}

Plugging this into the expression for $\phi(z)$,
\begin{align*}
\phi(z)&=-\frac{E_B}{e}\left(\left(\frac{z}{z_0}-1\right)^2-1\right)\\
&=-\left(\sqrt{\frac{2\pi N_de}{\varepsilon}}z-\sqrt{\frac{E_B}{e}}\right)^2+\frac{E_B}{e}
\end{align*}

Using the numerical values, we get
\begin{equation*}
z_0\approx5.76\times 10^{-7}\textup{ m}=5.76\times 10^3\textup{ \AA}
\end{equation*}

\subsection{Part (b)}
The electrons see a potential barrier
\begin{align*}
V(z)&=\eta(z)E_B-e\phi(B)\\
&=\left\{\begin{array}{ll}0&\quad\quad(z<0\textup{ or }z>z_0)\\E_B\left(\frac{z}{z_0}-1\right)^2&\quad\quad(0\leqslant z\leqslant z_0)\end{array}\right.
\end{align*}
where $\eta(z)=1$ if $z\geqslant 0$ and 0 otherwise.

Then
\begin{equation*}
\kappa(z)=\frac{\sqrt{2mV(z)}}{\hbar}=\frac{\sqrt{2mE_B}}{\hbar}\left(1-\frac{z}{z_0}\right)
\end{equation*}

And WKB gives
\begin{align*}
P&=\exp\left(-2\int_0^{z_0}\kappa(z)dz\right)\\
&=\exp\left(-\frac{2\sqrt{2mE_B}z_0}{\hbar}\int_0^1(1-x)dx\right)\\
&=\exp\left(-\frac{\sqrt{2mE_B}z_0}{\hbar}\right)\\
&=\exp\left(-\frac{E_B}{e\hbar}\sqrt{\frac{\varepsilon m}{\pi N_d}}\right)
\end{align*}

\subsection{Part (c)}
Probability for an electron to flow from the metal side to the semiconductor side is $P(m\rightarrow s)=(\exp(E_B/kT)+1)^{-1}\approx\exp(-E_B/kT)$. Probability for an electron to flow from the semiconductor side to the metal side is $P(s\rightarrow m)=(\exp((E_B-eV)/kT)+1)^{-1}\approx\exp((eV-E_B)/kT)$. Therefore, the current is
\begin{equation*}
I\propto P(s\rightarrow m)-P(m\rightarrow s)=\exp(-E_B/kT)(\exp(eV/kT)-1)
\end{equation*}

\end{document}
