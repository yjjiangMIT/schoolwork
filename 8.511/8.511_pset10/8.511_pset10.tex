\documentclass{article}
\usepackage{amsmath}
\usepackage{amssymb}
\usepackage{indentfirst}
\usepackage{graphicx}
\usepackage{color}
\usepackage{fancyhdr}
\usepackage{epstopdf}
\usepackage{indentfirst}
\usepackage{geometry}
\usepackage{bm}
\geometry{left=2.5cm,right=2.5cm,top=2.5cm,bottom=2.5cm}

\title{8.511 Problem Set 10}
\author{Yijun Jiang}
%\email{yjjiang@mit.edu}
\date{\today}

\pagestyle{fancy}
\lhead{Yijun Jiang}
\rhead{8.511 Problem Set 10}

\begin{document}
\maketitle
\section{Second Quantization}
\subsection{Part (a)}
A Hartree-Fock ket occupying states $\phi_{\alpha_1},\phi_{\alpha_2},\cdots,\phi_{\alpha_N}$ is described in second quantization by
\begin{equation*}
\Psi=\prod_ic_i^\dag|0\rangle
\end{equation*}

Since $\{c_{\alpha_i}^\dag,c_{\alpha_j}^\dag\}=0$ for any two creation operators, exchanging two creation operators changes the sign of $|\Psi\rangle$. This means that $|\Psi\rangle$ is antisymmetrized. Let $\hat{A}$ be the antisymmetrization operator, then
\begin{equation*}
\Psi=\hat{A}|\phi_{\alpha_1}\rangle|\phi_{\alpha_2}\rangle\cdots|\phi_{\alpha_N}\rangle
\end{equation*}

Therefore, in the representation of position and spin, the Hartree-Fock wavefunction is
\begin{align*}
\Psi(\mathbf{r}_1\sigma_1,\mathbf{r}_2\sigma_2,\cdots,\mathbf{r}_N\sigma_N)&=\langle\mathbf{r}_1\sigma_1|\langle\mathbf{r}_2\sigma_2|\cdots\langle\mathbf{r}_N\sigma_N|\Psi\rangle\\
&=\hat{A}\phi_{\alpha_1}(\mathbf{r}_1\sigma_1)\phi_{\alpha_2}(\mathbf{r}_2\sigma_2)\cdots\phi_{\alpha_N}(\mathbf{r}_N\sigma_N)
\end{align*}

The antisymmetrization operator $\hat{A}$ explores all permutations of $1,2,\cdots,N$ including the signs, giving exactly the following Slater determinant.
\begin{equation*}
\Psi(\mathbf{r}_1\sigma_1,\mathbf{r}_2\sigma_2,\cdots,\mathbf{r}_N\sigma_N)=\left|\begin{array}{cccc}\phi_{\alpha_1}(\mathbf{r}_1\sigma_1)&\phi_{\alpha_2}(\mathbf{r}_1\sigma_1)&\cdots&\phi_{\alpha_N}(\mathbf{r}_1\sigma_1)\\ \phi_{\alpha_1}(\mathbf{r}_2\sigma_2)&\phi_{\alpha_2}(\mathbf{r}_2\sigma_2)&\cdots&\phi_{\alpha_N}(\mathbf{r}_2\sigma_2)\\ \vdots&\vdots&\ddots&\vdots\\ \phi_{\alpha_1}(\mathbf{r}_N\sigma_N)&\phi_{\alpha_2}(\mathbf{r}_N\sigma_N)&\cdots&\phi_{\alpha_N}(\mathbf{r}_N\sigma_N)\end{array}\right|
\end{equation*}

\subsection{Part (b)}
Notice that for all $i$,
\begin{equation*}
c_i^\dag|\Psi\rangle=(-1)^{P_i}c_i^\dag c_i^\dag\prod_{j\neq i}c_j^\dag|0\rangle=0
\end{equation*}
where $P_i$ is the number of swaps to move $c_i$ to the beginning of all other creation operators, and the identity $c_i^\dag c_i^\dag=0$ is used. Therefore,
\begin{align*}
\langle\Psi|H_0|\Psi\rangle&=\sum_{ij}f_{ij}\langle\Psi|c_i^\dag c_j|\Psi\rangle\\
&=\sum_{ij}f_{ij}\langle\Psi|(\delta_{ij}-c_jc_i^\dag)|\Psi\rangle\\
&=\sum_if_{ii}\langle\Psi|\Psi\rangle-\sum_{ij}f_{ij}\langle\Psi|c_jc_i^\dag|\Psi\rangle\\
&=\sum_if_{ii}
\end{align*}

Moreover,
\begin{align*}
\langle\Psi|H_1|\Psi\rangle&=\sum_{ijkl}V_{ijkl}\langle\Psi|c_i^\dag c_j^\dag c_kc_l|\Psi\rangle\\
&=\sum_{ijkl}V_{ijkl}\langle\Psi|c_i^\dag(\delta_{jk}-c_kc_j^\dag)c_l|\Psi\rangle\\
&=\sum_{ijkl}V_{ijkl}\langle\Psi|\delta_{jk}(\delta_{il}-c_lc_i^\dag)-(\delta_{ik}-c_kc_i^\dag)(\delta_{jl}-c_lc_j^\dag)|\Psi\rangle\\
&=\sum_{ijkl}V_{ijkl}\langle\Psi|\delta_{il}\delta_{jk}-\delta_{jk}c_lc_i^\dag-\delta_{ik}\delta_{jl}+\delta_{jl}c_kc_i^\dag+\delta_{ik}c_lc_j^\dag-c_kc_i^\dag c_lc_j^\dag|\Psi\rangle\\
&=\sum_{ijkl}V_{ijkl}\langle\Psi|\delta_{il}\delta_{jk}-\delta_{ik}\delta_{jl}|\Psi\rangle\\
&=\sum_{ij}V_{ijji}-\sum_{ij}V_{ijij}\\
&=\sum_{i\neq j}V_{ijji}-\sum_{i\neq j}V_{ijij}\\
&=\sum_{i\neq j}v_{ijji}\langle\chi_i|\chi_i\rangle\langle\chi_j|\chi_j\rangle-\sum_{i\neq j}v_{ijij}\langle\chi_i|\chi_j\rangle\langle\chi_j|\chi_i\rangle\\
&=\sum_{i\neq j}v_{ijji}-\sum_{i\neq j,\textup{spin}\parallel}v_{ijij}
\end{align*}

Therefore,
\begin{equation*}
\langle\Psi|H|\Psi\rangle=\langle\Psi|H_0|\Psi\rangle+\langle\Psi|H_1|\Psi\rangle=\sum_if_{ii}+\sum_{i\neq j}v_{ijji}-\sum_{i\neq j,\textup{spin}\parallel}v_{ijij}
\end{equation*}

Write in first quantization language,
\begin{align*}
\langle\Psi|H_0|\Psi\rangle&=\sum_if_{ii}\\
&=\sum_i\int d\mathbf{r}_1u_i^*(1)\left(\-\frac{\hbar^2\nabla_1^2}{2m}+\frac{Ze^2}{r_1}\right)u_i(1)\langle\chi_i|\chi_i\rangle\\
&=\sum_i\int d\mathbf{r}_1u_i^*(1)h(1)u_i(1)
\end{align*}
where $h(1)$ is the one body operator of kinetic energy and external potential.

\begin{align*}
\langle\Psi|H_1|\Psi\rangle&=\sum_{i\neq j}v_{ijji}-\sum_{i\neq j,\textup{spin}\parallel}v_{ijij}\\
&=\frac{1}{2}\int d\mathbf{r}_1u_i^*(1)\left(\sum_{j,j\neq i}\int d\mathbf{r}_2u_j^*(2)\frac{e^2}{r_{12}}u_j(2)\right)u_i(1)-\frac{1}{2}\int d\mathbf{r}_1u_i^*(1)\sum_{j,j\neq i,\textup{spin}\parallel}\int d\mathbf{r}_2u_j^*(2)\frac{e^2}{r_{12}}u_i(2)u_j(1)\\
&=\frac{1}{2}\sum_i\int d\mathbf{r}_1u_i^*(1)V_H(1)u_i(1)+\frac{1}{2}\sum_i\int d\mathbf{r}_1u_i^*(1)\hat{V}_{ex}(1)u_i(1)\\
\end{align*}

Therefore,
\begin{align*}
\langle\Psi|H|\Psi\rangle&=\langle\Psi|H_0|\Psi\rangle+\langle\Psi|H_1|\Psi\rangle\\
&=\sum_i\int d\mathbf{r}_ 1u_i^*(1)h(1)u_i(1)+\frac{1}{2}\sum_i\int d\mathbf{r}_1u_i^*(1)V_H(1)u_i(1)+\frac{1}{2}\sum_i\int d\mathbf{r}_1u_i^*(1)\hat{V}_{ex}(1)u_i(1)\\
&=\sum_i\int d\mathbf{r}_1u_i^*(1)(h(1)+V_H(1)+\hat{V}_{ex}(1))u_i(1)-\frac{1}{2}\sum_i\int d\mathbf{r}_1u_i^*(1)V_H(1)u_i(1)-\frac{1}{2}\sum_i\int d\mathbf{r}_1u_i^*(1)\hat{V}_{ex}(1)u_i(1)\\
&=\sum_i\varepsilon_i-\frac{1}{2}\sum_i\int d\mathbf{r}_1u_i^*(1)V_H(1)u_i(1)-\frac{1}{2}\sum_i\int d\mathbf{r}_1u_i^*(1)\hat{V}_{ex}(1)u_i(1)
\end{align*}
which gives exactly what we have obtained using first quantization.

\section{Local Density Approximation}
\subsection{Part (a)}
The Hohenberg-Kohn energy functional is
\begin{equation*}
E^{(H\!K)}[n(\mathbf{r})]=T[n(\mathbf{r})]+V_{ee}[n(\mathbf{r})]+\int V_{ext}(\mathbf{r})n(\mathbf{r})d\mathbf{r}
\end{equation*}
where $T[n(\mathbf{r})]$ is the kinetic energy, $V_{ee}[n(\mathbf{r})]$ the electron-electron interactions, and $V_{ext}(\mathbf{r})$ the external potential. Imagine a non-interacting electron system with the same density $n(\mathbf{r})$, which can be decomposed as $n(\mathbf{r})=\sum_i\psi_i^*(\mathbf{r})\psi_i(\mathbf{r})$. Rewrite the energy functional as

\begin{align*}
E^{(H\!K)}[n(\mathbf{r})]&=T_0[n(\mathbf{r})]+V_H[n(\mathbf{r})]+\int V_{ext}(\mathbf{r})n(\mathbf{r})d\mathbf{r}+E_{xc}[n(\mathbf{r})]\\
&=\sum_i\int\psi_i^*(\mathbf{r})\left(-\frac{\hbar^2\nabla^2}{2m}+V_{ext}\right)\psi_i(\mathbf{r})d\mathbf{r}+\frac{1}{2}\sum_{ij}\int\psi_i^*(\mathbf{r})\psi_j^*(\mathbf{r}')\frac{e^2}{|\mathbf{r}-\mathbf{r}'|}\psi_i(\mathbf{r})\psi_j(\mathbf{r}')d\mathbf{r}d\mathbf{r}'+E_{xc}[n(\mathbf{r})]
\end{align*}
where $E_{xc}[n(\mathbf{r})]=T[n(\mathbf{r})]+V_{ee}[n(\mathbf{r})]-T_0[n(\mathbf{r})]-V_H[n(\mathbf{r})]$.

Defining Lagrange multipliers $\varepsilon_{ij}$, we want to make the variation of $E^{(H\!K)}[n(\mathbf{r})]-\varepsilon_{ij}(\int\psi_i^*(\mathbf{r})\psi_j(\mathbf{r})d\mathbf{r}-1)$ vanish. Treat $\psi_i(\mathbf{r})$ and $\psi_i^*(\mathbf{r})$ as independent, and pick orthogonal orbitals such that $\varepsilon_{ij}=\varepsilon_i\delta_{ij}$. Therefore,
\begin{align*}
&\int\delta\psi_i^*(\mathbf{r})\left(-\frac{\hbar^2\nabla^2}{2m}+V_{ext}\right)\psi_i(\mathbf{r})d\mathbf{r}+\sum_j\int\delta\psi_i^*(\mathbf{r})\psi_j^*(\mathbf{r}')\frac{e^2}{|\mathbf{r}-\mathbf{r}'|}\psi_i(\mathbf{r})\psi_j(\mathbf{r}')d\mathbf{r}d\mathbf{r}'+\int V_{xc}[n(\mathbf{r})]\delta\psi_i^*(\mathbf{r})\psi_i(\mathbf{r})d\mathbf{r}\\
=&\varepsilon_i\int\delta\psi_i^*(\mathbf{r})\psi_i(\mathbf{r})d\mathbf{r}
\end{align*}
where $V_{xc}[n(\mathbf{r})]$ is the functional derivative of $E_{xc}[n(\mathbf{r})]$. In Local Density Approximation, $E_{xc}[n(\mathbf{r})]=\int n(\mathbf{r})\varepsilon_{xc}(n(\mathbf{r}))d\mathbf{r}$, and $V_{xc}(n(\mathbf{r}))=\mu_{xc}(n(\mathbf{r}))=d(n\varepsilon_{xc})/dn$.

Since $\delta\psi_i^*(\mathbf{r})$ is arbitrary, it follows that
\begin{equation*}
\left(-\frac{\hbar^2\nabla^2}{2m}+V_{ext}+\int\frac{n(\mathbf{r}')e^2}{|\mathbf{r}-\mathbf{r}'|}d\mathbf{r}'+\mu_{xc}(n(\mathbf{r}))\right)\psi_i(\mathbf{r})=\varepsilon_i\psi_i(\mathbf{r})
\end{equation*}

Therefore,
\begin{align*}
\sum_i\varepsilon_i&=\left(\sum_i\int\psi_i^*(\mathbf{r})\left(-\frac{\hbar^2\nabla^2}{2m}\right)\psi_i^*(\mathbf{r})d\mathbf{r}\right)+\int V_{ext}(\mathbf{r})n(\mathbf{r})d\mathbf{r}+\int\frac{n(\mathbf{r})n(\mathbf{r}')e^2}{|\mathbf{r}-\mathbf{r}'|}d\mathbf{r}d\mathbf{r}'+\int\mu_{xc}(n(\mathbf{r}))n(\mathbf{r})d\mathbf{r}\\
&=T_0+2V_H+\int V_{ext}(\mathbf{r})n(\mathbf{r})d\mathbf{r}+\int\mu_{xc}(n(\mathbf{r}))n(\mathbf{r})d\mathbf{r}
\end{align*}

Comparing with the expression for $E^{(H\!K)}$,
\begin{align*}
E^{(H\!K)}&=\sum_i\varepsilon_i-V_H+E_{xc}-\int\mu_{xc}(n(\mathbf{r}))n(\mathbf{r})d\mathbf{r}\\
&=\sum_i\varepsilon_i-\frac{1}{2}\int\frac{n(\mathbf{r})n(\mathbf{r}')e^2}{|\mathbf{r}-\mathbf{r}'|}d\mathbf{r}d\mathbf{r}'+\int(\varepsilon_{xc}(n(\mathbf{r}))-\mu_{xc}(n(\mathbf{r})))n(\mathbf{r})d\mathbf{r}
\end{align*}

\subsection{Part (b)}


\end{document}
