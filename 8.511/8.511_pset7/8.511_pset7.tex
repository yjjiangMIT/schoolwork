\documentclass{article}
\usepackage{amsmath}
\usepackage{amssymb}
\usepackage{indentfirst}
\usepackage{graphicx}
\usepackage{color}
\usepackage{fancyhdr}
\usepackage{epstopdf}
\usepackage{indentfirst}
\usepackage{geometry}
\usepackage{bm}
\geometry{left=2.5cm,right=2.5cm,top=2.5cm,bottom=2.5cm}

\title{8.511 Problem Set 7}
\author{Yijun Jiang}
%\email{yjjiang@mit.edu}
\date{\today}

\pagestyle{fancy}
\lhead{Yijun Jiang}
\rhead{8.511 Problem Set 7}

\begin{document}
\maketitle

\section{Graphene in a Magnetic Field}
\subsection{Part (a)}
Letting $|H-EI|=0$, we have
\begin{equation*}
E^2=(\hbar v)^2(k_x-ik_y)(k_x+ik_y)=(\hbar v|\mathbf{k}|)^2
\end{equation*}

Therefore, we recover the linear spectrum
\begin{equation*}
E_{\mathbf{k}}^{\pm}=\pm\hbar v|\mathbf{k}|
\end{equation*}

\subsection{Part (b)}
\begin{align*}
\pi_x&=-i\partial_x+\frac{e}{\hbar c}A_x=-i\partial_x\\
\pi_y&=-i\partial_y+\frac{e}{\hbar c}A_y=-i\partial_y+\frac{eB}{\hbar c}x
\end{align*}

Therefore, the Schr\"{o}dinger equation is
\begin{equation*}
\hbar v\left(\begin{array}{cc}0&-i\partial_x-\partial_y-i\frac{eB}{\hbar c}x\\-i\partial_x+\partial_y+i\frac{eB}{\hbar c}x&0\end{array}\right)\left(\begin{array}{c}\phi_1\\ \phi_2\end{array}\right)=E\left(\begin{array}{c}\phi_1\\ \phi_2\end{array}\right)
\end{equation*}

Acting by $H$ on both sides, we have
\begin{equation*}
(\hbar v)^2\left(\begin{array}{cc}(-i\partial_x-\partial_y-i\frac{eB}{\hbar c}x)(-i\partial_x+\partial_y+i\frac{eB}{\hbar c}x)&0\\0&(-i\partial_x+\partial_y+i\frac{eB}{\hbar c}x)(-i\partial_x-\partial_y-i\frac{eB}{\hbar c}x)\end{array}\right)\left(\begin{array}{c}\phi_1\\ \phi_2\end{array}\right)=E^2\left(\begin{array}{c}\phi_1\\ \phi_2\end{array}\right)
\end{equation*}
which simplifies to
\begin{equation*}
(\hbar v)^2\left(\begin{array}{cc}-\partial_{xx}+(i\partial_y-\frac{eB}{\hbar c}x)^2+\frac{eB}{\hbar c}[\partial_x,x]&0\\0&-\partial_{xx}+(i\partial_y-\frac{eB}{\hbar c}x)^2-\frac{eB}{\hbar c}[\partial_x,x]\end{array}\right)\left(\begin{array}{c}\phi_1\\ \phi_2\end{array}\right)=E^2\left(\begin{array}{c}\phi_1\\ \phi_2\end{array}\right)
\end{equation*}

Since $[\partial_x,x]=1$, the equation for $\phi_2$ is thus
\begin{equation*}
(\hbar v)^2\left(-\partial_{xx}+\left(i\partial_y-\frac{eB}{\hbar c}x\right)^2-\frac{eB}{\hbar c}\right)\phi_2=E^2\phi_2
\end{equation*}

\subsection{Part (c)}
The $y$ dependence only enters by a phase $e^{ik_yy}$. Replacing $\partial_y\rightarrow ik_y$,
\begin{equation*}
(\hbar v)^2\left(-\partial_{xx}+\left(k_y+\frac{eB}{\hbar c}x\right)^2-\frac{eB}{\hbar c}\right)f(x)=E^2f(x)
\end{equation*}

This is just a harmonic oscillator equation,
\begin{equation*}
\left(-\frac{\hbar^2}{2m}\partial_{xx}+\frac{1}{2}m\omega^2\left(x+\frac{\hbar c}{eB}k_y\right)^2\right)f(x)=E'f(x)
\end{equation*}
where
\begin{align*}
&\omega=\frac{eB}{mc}\\
&E'=\frac{E^2}{2mv^2}+\frac{1}{2}\hbar\omega
\end{align*}

The eigenvalues are
\begin{equation*}
E'_n=\left(n+\frac{1}{2}\right)\hbar\omega\quad\quad(n=0,1,2\cdots)
\end{equation*}
which means
\begin{equation*}
E^2_n=2nmv^2\hbar\omega=\frac{2n\hbar eBv^2}{c}\quad\quad(n=0,1,2\cdots)
\end{equation*}

So we get the Landau levels (LL hereafter)
\begin{equation*}
E_n^{\pm}=\pm v\sqrt{\frac{2e\hbar}{c}Bn}\quad\quad(n=0,1,2\cdots)
\end{equation*}

Plugging in numerical values,
\begin{equation*}
E_1^+\approx9.18\times10^{-2}\textup{ eV}\sim10^3\textup{ K}
\end{equation*}

This gap is much larger than LL spacing for a free 2DEG under the same magnetic field,
\begin{equation*}
\Delta E_{free}=\frac{\hbar eB}{mc}\approx1.16\times10^{-3}\textup{ eV}\sim13.4\textup{ K}
\end{equation*}

\subsection{Part (d)}
The degeneracy comes from three factors. (1) There are two inequivalent Dirac cones. (2) Spin degeneracy. Zeeman effect is on the order of $\mu_BB\sim10^{-4}$ eV for $B\sim10$ T, which is much smaller than LL spacing. So we think of spin degeneracy as not being lifted. (3) $E_n^{\pm}$ does not depend on $k_y$.

Since the sample is of finite size, $k_y$ takes discrete values $k_y=\frac{2\pi N'}{L_y}$. Moreover, $k_y$ is bounded, for the localization in $x$ direction must stay within the sample.
\begin{equation*}
\frac{\hbar c}{eB}(k_{y,\max}-k_{y,\min})=\frac{h c}{eBL_y}N'=L_x
\end{equation*}

Therefore,
\begin{equation*}
N'=\frac{eBL_xL_y}{hc}=\frac{AB}{\phi_0}
\end{equation*}
where $A$ is the size of the sample. Considering two inequivalent Dirac cones and two spins, the degeneracy is
\begin{equation*}
N=4N'=\frac{4AB}{\phi_0}
\end{equation*}

\subsection{Part (e)}
Suppose $n$ LLs are filled at field intensity $B$. Since LL degeneracy is linear in $B$, $cnB$ electrons are accommodated, where $c$ is a porprotional constant. This means that
\begin{equation*}
\int_0^{E_F}g(\varepsilon)d\varepsilon=cnB
\end{equation*}

The fact $g(\varepsilon)\propto\varepsilon$ implies that $E_F\propto\sqrt{nB}$. Since $E_F=E_n$, we conclude that LLs scale with $\sqrt{B}$.

\section{Shubnikov-de Haas Oscillation}
\subsection{Part (a)}
In SI units, Shubnikov-de Haas relation gives
\begin{equation*}
\Delta\left(\frac{1}{B}\right)=\frac{2\pi e}{\hbar S_F}
\end{equation*}
where $S_F$ is the extremal cross-sectional area of the Fermi surface. In free electron model, the Fermi surface is a sphere whose radius is $k_F$. To calculate $k_F$, we have
\begin{equation*}
\frac{1}{V}\frac{\frac{4}{3}\pi k_F^3\times 2}{\left(\frac{2\pi}{L}\right)^3}=n
\end{equation*}

Therefore, $k_F=\sqrt[3]{3\pi^2n}$, where $n$ is the spatial density of electrons. Since Na has bcc lattice and is of valence 1,
\begin{equation*}
n=\frac{2}{a^3}=\frac{2}{(429.06\textup{ pm})^3}=2.532\times 10^{28}\textup{ m}^{-3}
\end{equation*}

Therefore, $k=9.08\times 10^9\textup{ m}^{-1}$, and the period is
\begin{equation*}
\Delta\left(\frac{1}{B}\right)=\frac{2e}{\hbar k_F^2}=3.68\times 10^{-5}\textup{ T}^{-1}
\end{equation*}

\subsection{Part (b)}
$B=10\textup{ T}$ corresponds to
\begin{equation*}
n=\frac{\frac{1}{B}}{\Delta\left(\frac{1}{B}\right)}\approx 2716
\end{equation*}

Therefore, the area enclosed by the real space orbital is
\begin{equation*}
A=\frac{\Phi}{B}=\frac{n\Phi_0}{B}=\frac{nh}{eB}=1.12\times 10^{-12}\textup{ m}^2
\end{equation*}

\end{document}
